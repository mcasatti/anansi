\ifdefined\ingles	

	In view of the results obtained during the elaboration of this work, it is considered that the graph databases are a valid mechanism for the storage of astronomical information, providing a flexible scheme but that at the same time can query efficiently large volumes of data.
	
	The labeled graphs allow to accurately reflect the astronomical data, also allowing the growth and adaptation of the structures according to the needs that arise from the data files, a task that is difficult in the case of using relational databases.
	
	In subsequent stages during the evolution of the thesis plan, it is expected to optimize the mechanisms for importing astronomical information and provide users with some intuitive tools to perform both the data import tasks and the definition of the criteria for the generation of relationships demand.
	
	An end user query system will be developed to indicate the parameters to be used in the searches and a module will be implemented that obtains some relevant statistical indicators according to the stored data.
	
	Subsequently, some pattern models existing in other disciplines will be analyzed and tests will be implemented to use these patterns in an astronomical environment. The purpose of this activity is to determine if there are pre-existing patterns that have application in an area for which they have not been designed. Of special interest are the small-world network algorithms\cite{kleinberg2000navigation} and the social network algorithms\cite{carrington2005models}.

\else

	En vista a los resultados obtenidos durante la elaboración del presente trabajo se considera que las bases de datos de grafos son un mecanismo válioso para el almacenamiento de información astronómica, brindando un esquema flexible pero que a la vez puede realizar consultas de manera eficiente en grandes volúmenes de datos.
	
	Los grafos etiquetados permiten reflejar con exactitud los datos astronómicos, permitiendo asimismo el crecimiento y adecuación de las estructuras de acuerdo a las necesidades que surjan a partir de los archivos de datos, tarea ésta que es dificultosa en el caso de utilizar bases de datos relacionales.
	
	En posteriores etapas durante la evolución del plan de tesis se prevé optimizar los mecanismos de importación de información astronómica y brindar a los usuarios algunas herramientas intuitivas para realizar tanto las tareas de importación de datos como la definición de los criterios para la generación de las relaciones a demanda.
	
	Se desarrollará un sistema de consulta que permita indicar los parámetros a utilizar en las búsquedas y se implementará un módulo que obtenga algunos indicadores estadísticos relevantes de acuerdo a los datos almacenados.
	
	Posteriormente se analizarán algunos modelos de patrones existentes en otras disciplinas y se implementarán tests para utilizar dichos patrones en un ámbito astronómico. La finalidad de ésta actividad es determinar si existen patrones pre-existentes que tengan aplicación en un ámbito para el que no han sido diseñados. Un especial interés tienen los algoritmos de redes de mundo pequeño\cite{kleinberg2000navigation} y los algoritmos de redes sociales\cite{carrington2005models}.

\fi