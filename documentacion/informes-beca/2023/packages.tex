%\usepackage[]{fontspec}
%\setmainfont{Times New Roman}
\usepackage[]{newtx}
\usepackage[spanish]{babel}

% ======================================================
% GESTION DE COLORES Y GRAFICOS
% ======================================================
\usepackage[dvipsnames,svgnames]{xcolor}
\usepackage{graphicx}
% ======================================================
% MEJORA LA LEGIBILIDAD GENERAL DEL DOCUMENTO Y LAYOUT
% ======================================================
% Microtype tiene que preceder a Ragged2e
\usepackage[babel=true,expansion=true,protrusion=true]{microtype}
% Ragged2e tiene que ir despues de microtype para funcionar correctamente sino cambia las alineaciones y no quedan bien
\usepackage{ragged2e}

% ======================================================
% MARCAS DE AGUA
% ======================================================
\usepackage{draftwatermark}
% ======================================================
% PERMITE INCORPORAR HIPERVINCULOS (internos y URL externas)
% ======================================================
\usepackage[
	colorlinks,
	urlcolor = blue,
	linkcolor = black,
	citecolor = blue
	%pdfborder = {0 0 0}
]{hyperref}


% ======================================================
% MEJORES LISTAS DE ENUMERACION
% ======================================================
\usepackage{enumitem}
% Para poder tachar texto con \uwave{}, uwave{}, sout{}
\usepackage[normalem]{ulem} % El parametro normalem es para no alterar el emphasis (italica)

% ======================================================
% TODOLIST
% ======================================================
\setlength{\marginparwidth}{2cm}
\usepackage[colorinlistoftodos,textsize=tiny]{todonotes}
\ifdefined\borrador
	\newcommand{\revisar}[1]{\todo[backgroundcolor=yellow!50]{\textbf{Revisar:} #1}}
	\newcommand{\ampliar}{\todo[backgroundcolor=blue!30]{Ampliar}}
	\newcommand{\revisarcita}{\colorbox{red}{\textcolor{white}{\ CITA\ }}}
	\newcommand{\sincita}{\colorbox{red}{\textcolor{white}{\ CITA\ }}}
	\newcommand{\faltafigura}{\missingfigure{Falta figura\ldots}}
	\newcommand{\citaurl}{\todo[backgroundcolor=green!50]{URL}}
\else
	\newcommand{\revisar}[1]{}
	\newcommand{\ampliar}{}
	\newcommand{\revisarcita}{}
	\newcommand{\sincita}{}
	\newcommand{\faltafigura}{}
	\newcommand{\citaurl}{}
\fi

\usepackage{csquotes}

% ======================================================
% GANTT
% ======================================================
\usepackage{pgfgantt}
% ======================================================
% FLOATS y REORDENAMIENTO DE TEXTO ALREDEDOR DE FIGURAS
% ======================================================
\usepackage{float}
\usepackage{wrapfig}