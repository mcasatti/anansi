\documentclass[
	11pt,oneside,a4paper,
	headsepline,footsepline,
	%plainfootsepline,plainheadsepline,
	fleqn,
	%flushbottom,
	%raggedbottom
]{memoir}
%]{scrartcl}
%]{article}
%\usepackage[a4paper, total={6in, 8in}]{geometry}

%------------------------------------------
% FLAG DE BORRADOR
%------------------------------------------
\def\borrador{}
%------------------------------------------

\setlength {\marginparwidth }{2cm}

% Required for specifying colors by name
\usepackage[usenames,dvipsnames,svgnames,table]{xcolor} 
% Define the orange color used for highlighting throughout the book
\definecolor{ocre}{RGB}{243,102,25} 
\usepackage[export]{adjustbox}
\usepackage{subcaption}
\usepackage{float}
% Permite utilizar encabezados y pies de pagina simples con KOMA-Script
\usepackage{scrlayer-scrpage}
% Page margins
\usepackage[top=3cm,bottom=3cm,left=3cm,right=3cm,headsep=10pt,a4paper]{geometry} 
\usepackage[document]{ragged2e}
% Required for including pictures
\usepackage{graphicx} 
% Inserts dummy text
\usepackage{lipsum} 
%\usepackage[english]{babel} % English language/hyphenation
\usepackage[spanish,es-noquoting]{babel} % English language/hyphenation
% Customize lists
\usepackage{enumitem} 
% Resaltado de sintaxis en listados de código fuente
\usepackage{listings}
% Required for nicer horizontal rules in tables
\usepackage{booktabs} 
% Mejor manejo de tablas
\usepackage{tabularx} 
% Manejo de hipervinculos
\usepackage[hidelinks]{hyperref}
%Manejo de referencias y bibliografía
%\usepackage
%[
%	%style=alphabetic,
%	style=numeric,
%	citestyle=numeric,
%	sorting=none,
%	%sortcites=true,
%	%autopunct=true,
%	babel=hyphen,
%	hyperref=true,
%	abbreviate=false,
%	backref=true,
%	%backend=biblatex,
%	defernumbers=false
%]
%{biblatex}
% Permite sacar parte del codigo a un archivo externo. Util para referenciar bibliografía embebida
%\usepackage{filecontents}
%----------------------------------------------------------------------------------------
%	ALGORITMOS
%----------------------------------------------------------------------------------------
\usepackage{algorithm2e}
% Slightly tweak font spacing for aesthetics
%\usepackage{microtype} 
% Required for including letters with accents
\usepackage[utf8]{inputenc} 
% Use 8-bit encoding that has 256 glyphs
\usepackage[T1]{fontenc} 
% Agregar fuentes monoespaciadas bold
\usepackage{bold-extra}
% Using Courier font
%\renewcommand{\ttdefault}{pcr}
%----------------------------------------------------------------------------------------
%	DEFINICION DE DIVERSOS TIPOS DE TO-DO
%----------------------------------------------------------------------------------------
\usepackage[colorinlistoftodos,prependcaption,textsize=tiny]{todonotes}
\usepackage{tablefootnote}
\usepackage{xargs}

\newcommandx{\revisar}[2][1=]{\todo[linecolor=red,backgroundcolor=red!25,bordercolor=red,#1]{#2}}
\newcommandx{\cambiar}[2][1=]{\todo[linecolor=blue,backgroundcolor=blue!25,bordercolor=blue,#1]{#2}}
\newcommandx{\info}[2][1=]{\todo[linecolor=green,backgroundcolor=green!25,bordercolor=green,#1]{#2}}
\newcommandx{\mejorar}[2][1=]{\todo[linecolor=gray,backgroundcolor=Plum!25,bordercolor=gray,#1]{#2}}
\newcommandx{\faltafigura}[2][1=]{\missingfigure[#1,figcolor=white]{#2}}
\newcommandx{\nomostrar}[2][1=]{\todo[disable,#1]{#2}}

% Package para creacion de grafos simplificados
\usepackage{tkz-graph}
% Required for drawing custom shapes
\usepackage{tikz} 
\usetikzlibrary{
	arrows,
	decorations.pathmorphing,
	backgrounds,
	positioning,
	fit,
	petri,
	quotes,
	% Este paquete produce errores usado en conjunto con tkz-graph
	%babel,	
	arrows.meta,
	decorations.pathreplacing,
	shapes
}

%----------------------------------------------------------------------------------------
%	CONFIGURACION DE LOS LISTADOS DE CODIGO FUENTE
%----------------------------------------------------------------------------------------
\definecolor{armygreen}{rgb}{0.29, 0.33, 0.13}
\lstset{language=Java,
	tabsize=3,
	showstringspaces=false,
	numbers=left,
	basicstyle=\ttfamily,
	keywordstyle=\color{black}\ttfamily\bfseries,
	stringstyle=\color{armygreen}\ttfamily,
	commentstyle=\color{gray}\ttfamily,
	morecomment=[l][\color{gray}]{\#}
}

\usepackage{csquotes}

\usepackage[
	backend=biber,
	%sorting=anyt,  %% Alfabetico
	sorting=none,	%% En orden de aparición
	style=numeric,
	citestyle=numeric
]{biblatex}
\DeclareBibliographyCategory{cited}
\AtEveryCitekey{\addtocategory{cited}{\thefield{entrykey}}}
%\addbibresource{bibliografia.bib}
\addbibresource{refs_used.bib}
\nocite{*}
\BiblatexSplitbibDefernumbersWarningOff


\ifdefined\borrador
	\SetWatermarkText{Borrador}
	\SetWatermarkScale{0.8}
	\SetWatermarkLightness{0.9}
\else
	\SetWatermarkText{}
\fi

%------------------------------------------
% FUENTES
%------------------------------------------
%\usepackage[sfdefault,scaled=1]{FiraSans}
\usepackage[libertine,scaled=1]{newtx}
%------------------------------------------

\graphicspath{
	{figures/}
	{images}
}

\clearpairofpagestyles
%\ihead{\today}								% Header interno: Fecha
% ENCABEZADO
\ohead{\textit{Plan de Tesis}}		% Header interno: Titulo
\ihead{\raisebox{-7mm}{\includegraphics[height=10mm]{./images/utn-black}}}		% Footer interno: 
% PIE DE PAGINA
\ofoot{Página \pagemark}	% Header externo: Evento
%Titulo
%\ohead{\raisebox{-10mm}{\includegraphics[height=10mm]{./images/MISI3}}}

\Huge
\title{Detección de cumulos estelares en galaxias cercanas utilizando técnicas de Machine Learning y algoritmos de redes sociales}
%\date{\today}
\Large
\author{Esp. Ing. Martin Casatti\\
\\Universidad Tecnológica Nacional\\
Facultad Regional Córdoba}

% Impedir que se comience a numerar a partir de capitulos (como el plan no tiene capitulos los numeros saldrían como 0.1, 0.2, etc)
\counterwithout{section}{chapter}

% INTERLINEADO 1 1/2 
%\OnehalfSpacing

\pgfplotsset{compat=1.18}

\begin{document}

\maketitle
\revisar{revisar la parte de ``algoritmos de redes sociales''}
\revisar{Agregar Carrera y Facultad. Indicar Director y CoDirector}
\revisar{Controlar extensión del documento. 8 páginas. (incluye caratula?)}
\revisar{Controlar las extensiones de las secciones también}
\revisar{Replantear página de portada como página normal, para controlar mejor el formato}


\newpage

\tableofcontents

\listoftodos

\newpage

\section {Introducción}

\subsection{Detección e identificación de cúmulos estelares}

Las agrupaciones estelares, también denominados clusters, han sido objetos reconocidos desde hace tiempo como laboratorios importantes para la investigación astrofísica. Estos objetos son muy útiles en varios aspectos, entre los que se pueden destacar los siguientes:

\begin{itemize}
	\item Los cúmulos estelares contienen muestras estadísticamente significativas de estrellas de aproximadamente la misma edad, con composiciones químicas similares, con un amplio rango de masas estelares y localizadas en un volumen relativamente pequeño del espacio, haciéndolas un conjunto ideal para el análisis de características comunes y determinación de los patrones que rigen su surgimiento \cite{Klessen2000}.
	\item En relación con el proceso de formación estelar, los cúmulos jóvenes permiten esclarecer la forma y las escalas de tiempo en las que estos mecanismos están activos \cite{Sung1998}, así como también permiten analizar su dependendencia de los distintos ambientes interestelares de la Vía Láctea o de otras galaxias \cite{Fall2012}.	
\end{itemize}

Los trabajos mencionados se han focalizado en mejorar el conocimiento de nuestra propia Galaxia (y de las Nubes de Magallanes\cite{Vazquez2008}), pero actualmente hay varios factores que incrementan de forma importante tanto la cantidad de objetos a investigar cómo la metodología para hacerlo.

En la actualidad existe una gran cantidad de información de las galaxias cercanas (a varios Mpc\footnote{Megaparsec, medida de distancia, aproximadamente 3.26 millones de años luz}) debido, en gran parte, a que el Telescopio Espacial Hubble (HST) ha permitido obtener datos con alta resolución espacial utilizando varias cámaras de campo amplio (WFPC2; ACS) \cite{Dalcanton2009}.

Se cuenta con una enorme cantidad de datos proveniente de las varias observaciones continuas que se están realizando y que se proyectan realizar en modo “survey”\footnote{Técnica que consiste en realizar un mapeo sistemático de una porción determinada de la esfera celeste sin concentrarse de manera puntual en ningún objeto.} (p.e. VVV, https://vvvsurvey.org/ o LSST; https://www.lsst.org/) que necesitan ser estudiados con métodos automáticos. En particular para la identificación y parametrización de nuevas agrupaciones estelares.

Los algoritmos de reconocimiento automático de patrones, aplicados a la búsqueda de agrupaciones estelares, están en la actualidad teniendo una importante revisión y desarrollo, dado que su uso es vital para el análisis de los “surveys” que se están realizando. Puede encontrarse en Schmeja,(2011) \cite{Schmeja2011} un análisis comparativo detallado de los diferentes métodos y sus técnicas.

Tal como se desprende de esa publicación, estos algoritmos se basan en analizar sólo las posiciones espaciales para encontrar a los sistemas estelares por sobre-densidades contra el fondo estelar o por su equivalente relacionado con la distribución de distancias entre estrellas. 

Cabe hacer notar que ya se han desarrollado varios algoritmos que han sido aplicados con éxito en otros campos científicos. Entre estos se destacan algoritmos como “K-mean”, “Birch”, “Spectral Clustering”, “Dbscan”, etc.\cite{rodriguez2019clustering}

\subsection{El estudio de la estructura de redes sociales}

Por otra parte, el gran auge que tienen desde hace algunos años el análisis de redes sociales nos ha brindado otro amplio campo de estudios en el que se pueden apreciar algunos de los atributos que son comunes al problema de la detección de cumulos estelares, como por ejemplo:

\begin{itemize}
	\item En el ámbito de las redes sociales también se cuenta con una gran cantidad de datos
	\item Existe un conjunto de relaciones no evidentes entre los mismos y
	\item Un nutrido grupo de atributos analizables a fin de guiar la detección de patrones 
\end{itemize}

La estructura inherente de dichas redes es la de un grafo, dirigido o no, y sobre las mismas se pueden realizar multitud de análisis sustentados por la Teoría de Grafos \cite{West2001}, siendo este un campo ampliamente estudiado, tanto de manera analítica como algorítmica.

Diversos estudios, tanto de la topología de dichas redes \cite{Barnes1983} como de las características que presentan sus participantes, nos brindan un fértil campo para el estudio de algoritmos de detección de patrones estructurales, muchos de ellos asistidos por técnicas de Machine Learning \cite{Alharbi2021}.

En la actualidad el análisis de algoritmos y su aplicación para la determinación de las características de las redes sociales es un campo en permanente evolución.

Algoritmos como los de detección de comunidades\cite{wang2015review}, detección de anomalías\cite{kaur2016survey}, determinación de subredes similares, clustering dinámico\cite{boccaletti2007detecting} y predicción de enlaces más probables\cite{kushwah2016review}, son un ámbito en donde las técnicas de aprendizaje supervisado está encontrando cada vez más y mejores aplicaciones.

El entrenamiento de modelos específicos para la detección de este tipo de estructuras está dando lugar a cada vez más y mejores caracterizaciones de redes con una enorme cantidad de nodos y de relaciones, y abriendo el desarrollo a algoritmos más complejos y potentes.

Existen actualmente estudios comparativos de diversos algoritmos de detección de comunidades en redes \cite{PhysRevE.80.056117} que presentan resultados prometedores para la aplicación de dichos algoritmos, o derivaciones de los mismos, en ámbitos diferentes, tal como es el enfoque del presente trabajo.

\section {Justificación}

La puesta en funcionamiento de instrumentos de observación astronómica cada vez más potentes, durante los últimos 50 años, ha dado lugar a un crecimiento exponencial de la cantidad de objetos detectados, los que requieren análisis y estudio. Sin ir demasiado lejos, el recientemente lanzado telescopio James Webb produce casi 60 Gigabytes de información al día, la cual no puede ser almacenada de manera local y debe ser transmitida de inmediato el centro de control de misión\cite{webdata}.

El proyecto ``Legacy Survey of Space and Time'' (Rubin/LSST), basado en el observatorio Vera C. Rubin\footnote{\url{https://rubinobs.org/}}, en Chile, que se encuentra en las últimas etapas de construcción, se estima que producirá 20 TB (terabytes) de información cada noche, durante una vida útil de al menos 10 años\cite{Telescope2021Jul}.

Estos volúmenes de datos hacen que sea imprescindible la utilización de mecanismos automáticos para el análisis, lo que brinda una oportunidad inmejorable para el desarrollo y adecuación de algoritmos y la aplicación de técnicas avanzadas de Machine Learning provenientes de diversos dominios, en el ámbito astronómico.

Es en este sentido que creemos que los resultados del presente trabajo pueden aportar al avance de dichas técnicas y colaborar, en última instancia, en el avance científico y tecnológico.

%\section*{Objeto de estudio}

%El objeto de estudio, en particular, serán las galaxias espirales o irregulares, cercanas a la Vía Láctea, las cuales cuentan con una cantidad apreciable de estrellas azules, de gran importancia para la comunidad astronómica ya que son estrellas jóvenes en estadíos iniciales de evolución.

\section {Hipótesis de trabajo}

Es la intención de este trabajo de posgrado demostrar la viabilidad de la aplicación de técnicas diseñadas para la caracterización de redes sociales, en el ámbito de la astronomía, para la detección de cumulos estelares, aprovechando de esta manera los estudios existentes en la materia pero enfocados en un nuevo ámbito de aplicación.

Se postula que la aplicación de técnicas de machine learning para el entrenamiento de algoritmos inteligentes posibilitará que los algoritmos de detección y caracterización de comunidades en redes sociales, puedan detectar agrupaciones estelares, a partir del correspondiente cambio en los atributos descriptivos y estructurales, de acuerdo al nuevo ámbito de aplicación.

\section {Objetivos}

El presente trabajo tiene como finalidad demostrar la viabilidad de la utilización de técnicas algorítmicas de aplicación en el ámbito de redes sociales, específicamente las asociadas a comunidades de individuos, para la detección de agrupaciones estelares en galaxias cercanas.

Se analizará la viabilidad de dichas técnicas y se contrastarán los resultados obtenidos con respecto a los de otras técnicas, diseñadas específicamente para el ámbito astronómico, a fin de sacar conclusiones adecuadas al ámbito de aplicación específico.

A su vez se plantean los siguientes objetivos particulares a alcanzar:

\begin{itemize}
	\item Realizar una revisión sistemática del estado del arte en cuanto a algoritmos de detección de estructuras en el ámbito astronómico y de las redes sociales.
	\item Determinar la viabilidad de extrapolar algoritmos de uno de los ámbitos mencionados al otro, específicamente en lo que respecta a detección de estructuras determinadas, sobre estructuras de tipo grafo.
	\item Establecer los atributos mínimos necesarios para el entrenamiento de un algoritmo de detección asistido por machine learning.
	\item Obtener un modelo de machine learning confiable para la detección de estructuras estelares, en el ámbito específico de aplicación.
\end{itemize}

\section {Metodología}

Para alcanzar el objetivo final mencionado, se pretende realizar las siguientes acciones\revisar{Reemplazar acciones por actividades}:

\begin{itemize}
	\item Se analizarán las técnicas de reconocimiento de agrupaciones estelares existentes y sus resultados actuales.
	\item Se analizarán las técnicas de reconocimiento de comunidades en redes sociales y sus resultados actuales.
	\item Se determinarán los atributos entrenables por medio de técnicas de machine learning en el ámbito de las redes sociales y extrapolarlos al ámbito astronómico.
	\item Se modelará y entrenará un mecanismo de machine learning con los atributos astronómicos, ya sean mediciones reales o sus equivalentes simulados.
	\item Se utilizará el algoritmo, una vez entrenado, para detección de comunidades sobre muestras reales a fin de analizar su eficacia y eficiencia.
	\item Se elaborará un procedimiento general para el entrenamiento del algoritmo de detección y la aplicación de la técnica para su utilización en diferentes ámbitos astronómicos o con diferentes muestras.
\end{itemize}

\revisar{El formato sugerido es: \\
Qué se va a hacer\\
Con qué herramientas\\Qué se espera como resultado}

\revisar{Incluir la validación de la propuesta en la metodología}

\section{Resultados esperados}

\begin{itemize}
	\item Se espera, al concluir con el trabajo, contar con un modelo eficaz para la detección de agrupaciones estelares, en muestras de datos reales, con un grado de exactitud al menos comparable a los mecanismos actualmente utilizados en la comunidad astronómica para esa misma finalidad.
	\item Se espera demostrar que los algoritmos desarrollados para la detección de comunidades, sobre grafos de redes sociales, con las modificaciones pertinentes, pueden ser una buena alternativa a la detección de comunidades en un ámbito completamente diferente, como es el de las estrellas en galaxias cercanas.
	\item Se espera sentar las bases para el estudio continuo de técnicas no desarrolladas específicamente para el ámbito astronómico, pero de posible aplicación en el mismo.
	\item Se espera ayudar a la comunidad astronómica con una herramienta de simple implementación y que provea resultados valiosos, como complemento a las técnicas ya existentes.
\end{itemize}

\section {Cronograma y plan general de trabajo}

\begin{figure}[H]
	%	\begin{center}
	\begin{ganttchart}[
			hgrid,
			vgrid,
			% x unit=0.6cm,
			bar/.append style={draw=Black, fill=RoyalBlue!75},
			time slot format=isodate-yearmonth,
			time slot unit=month,
			newline shortcut=true,
			bar label node/.append style={align=right}
		]{2023-01}{2023-12}
		\gantttitlecalendar{year, month} \\
		\ganttbar{Estudio bibliográfico}{2023-01}{2023-06} \\
		\ganttbar{Análisis de técnicas\ganttalignnewline de detección de Clusters}{2023-03}{2023-06} \\
		\ganttbar{Análisis de técnicas\ganttalignnewline de detección de comunidades}{2023-05}{2023-08}\\
		\ganttbar{Definición y extrapolación\ganttalignnewline de atributos}{2023-06}{2023-08}\\
		\ganttbar{Elaboración de\ganttalignnewline modelo de ML}{2023-08}{2023-12}\\
	\end{ganttchart}
	\caption*{Primer año}
\end{figure}

\begin{figure}[H]
	%	\begin{center}
	\begin{ganttchart}[
			hgrid,
			vgrid,
			% x unit=0.6cm,
			bar/.append style={draw=Black, fill=RoyalBlue!75},
			time slot format=isodate-yearmonth,
			time slot unit=month,
			newline shortcut=true,
			bar label node/.append style={align=right}
		]{2024-01}{2024-12}
		\gantttitlecalendar{year, month} \\
		\ganttbar{Entrenamiento de\ganttalignnewline modelo de ML}{2024-01}{2024-03}\\
		\ganttbar{Prueba de\ganttalignnewline modelo de ML}{2024-03}{2024-05}\\
		\ganttbar{Detección de clusters\ganttalignnewline sobre datos astronómicos}{2024-05}{2024-08}\\
		\ganttbar{Analisis comparativo\ganttalignnewline contra otras técnicas}{2024-08}{2024-12}
		%\ganttbar{Algoritmos de\ganttalignnewline búsqueda}{2019-07}{2019-10}\\
		%\ganttbar{Análisis de métricas}{2019-10}{2019-11}\\
		%\ganttbar{Escritura de Tesis}{2019-09}{2019-11}\\
		%\ganttmilestone{Defensa de Tesis}{2019-12}
	\end{ganttchart}
	\caption*{Segundo año}
\end{figure}

\begin{figure}[H]
	%	\begin{center}
	\begin{ganttchart}[
			hgrid,
			vgrid,
			% x unit=0.6cm,
			bar/.append style={draw=Black, fill=RoyalBlue!75},
			time slot format=isodate-yearmonth,
			time slot unit=month,
			newline shortcut=true,
			bar label node/.append style={align=right}
		]{2025-01}{2025-12}
		\gantttitlecalendar{year, month} \\
		\ganttbar{Elaboración de Tesis}{2025-01}{2025-06}\\
		\ganttbar{Revisión de Tesis}{2025-06}{2025-09}\\
		\ganttmilestone{Defensa de Tesis}{2025-10}
	\end{ganttchart}
	\caption*{Tercer año}
\end{figure}

\section{Antecedentes del tesista}

En los últimos años se han realizado, en la unidad ejecutora, diversos estudios con respecto a la aplicación de grafos con diversos fines, entre ellos:

\begin{itemize}
	\item “Análisis cienciométrico de la producción en investigación científica y tecnológica en la Red de Ingeniería en Informática y sistemas de información de CONFEDI”. \\
	      Tipo de proyecto: UTN (PID UTN), \\
	      código identificador del proyecto: 7848, \\
	      Director: Roberto Muñoz. \\
	      Lugar: CIDS: Centro de Investigación, Desarrollo y Transferencia de Sistemas de Información. Facultad Regional Córdoba de la Universidad Tecnológica Nacional. \\
	      Período 2020 : En ejecución.
	\item “Análisis y detección de patrones en un grafo conceptual construido a partir de respuestas escritas en forma textual a preguntas sobre un tema específico : Fase II”. \\
	      Tipo de proyecto: UTN (PID UTN), \\
	      código identificador del proyecto: SIUTICO000778600, \\
	      Director: M. Alejandra Paz Menvielle. \\
	      Lugar: CIDS : Centro de Investigación, Desarrollo y Transferencia de Sistemas de Información. Facultad Regional Córdoba de la Universidad Tecnológica Nacional. \\
	      Período 2020 - 2021.
	\item “Análisis y detección de patrones en un grafo conceptual construido a partir de respuestas escritas en forma textual a preguntas sobre un tema específico”. \\
	      Tipo de proyecto: UTN (PID UTN), \\
	      código identificador del proyecto: SIUTNCO4812, \\
	      Director: M. Alejandra Paz Menvielle. \\
	      Lugar: CIDS : Centro de Investigación, Desarrollo y Transferencia de Sistemas de Información. Facultad Regional Córdoba de la Universidad Tecnológica Nacional. \\
	      Período 2018 - 2019.
	\item “Metodología para determinar la exactitud de una respuesta, escrita en forma textual, a un interrogante sobre un tema específico, aplicando herramientas informáticas”. \\
	      Tipo de proyecto: UTN (PID UTN), \\
	      código identificador del proyecto: EIUTNCO0003592, \\
	      Director: Mario Alberto Groppo. \\
	      Lugar: CIDS : Centro de Investigación, Desarrollo y Transferencia de Sistemas de Información. Facultad Regional Córdoba de la Universidad Tecnológica Nacional. \\
	      Período 2015-2017.
\end{itemize}

El postulante ha participado en estos proyectos en calidad de docente investigador, arquitecto de las soluciones, programador, co-autor de artículos y expositor en diversos congresos y encuentros científicos, algunos de los cuales se mencionan a continuación.

\begin{itemize}
	\item Text format written questions evaluation Methodology\cite{Menvielle2016}
	\item Caso de aplicación de representación del conocimiento utilizando grafos conceptuales en un sistema de corrección automatizado de exámenes\cite{PazMenvielle2017}
	\item Model and evaluation tool using graphs as knowledge base for the automated correction of exams in text format\cite{Menvielle2017}
	\item Análisis y detección de patrones en un grafo conceptual construido a partir de respuestas escritas en forma textual a preguntas sobre un tema específico\cite{PazMenvielle2018}
	\item Análisis cienciométrico de la producción en investigación científica y tecnológica\cite{Munoz2020}
\end{itemize}

\section{Aportes potenciales del trabajo}

\subsection{Contribución al avance del conocimiento científico y/o tecnológico}

El proyecto está focalizado en la elaboración de un modelo de detección de patrones estelares a partir de trabajos previos desarrollados sobre comunidades de redes sociales.

La demostración de viabilidad de dichas técnicas permitirá ampliar el espectro de herramientas utilizables para la detección de cumulos estelares y propiciará el estudio de la aplicación de técnicas similares en ámbitos diversos.

Asimismo permitirá establecer la validez de ciertas técnicas de detección de patrones en grafos de cualquier tipo sobre un conjunto de datos astronómicos.

\subsection{Contribución a la formación de recursos humanos}

El Ing. Martin Casatti se desempeña actualmente como docente investigador en el grupo dirigido por el Ing. Roberto Muñoz, Secretario Académico de UTN Facultad Regional Córdoba, que trabaja en el marco del Centro de Investigación, Desarrollo y Transferencia de Sistemas, dirigido por el Dr. Ing. Marcelo Marciszack, director propuesto para este trabajo de posgrado.

El trabajo de la presente propuesta se desarrollará en el marco de los grupos de investigación pertenecientes a dicho Centro.

\subsection{Transferencia prevista de los resultados, aplicaciones o conocimientos derivados del
	proyecto}

La transferencia de los resultados o conocimientos del proyecto se realizará, a nivel
profesional, por medio de publicaciones internacionales bajo referato, presentaciones en reuniones científicas y formación de recursos humanos.

Los resultados se compartirán y/o transferiran a instituciones relacionadas con la astronomía que estén interesadas en la aplicación de las técnicas aquí desarrolladas.

\section{Director y co-director del trabajo}

Para la realización del presente proyecto se postula como director y co-director, respectivamente, a los Dr. Marcelo Martín Marciszack y al Dr. Carlos Feinstein Baigorri, de los cuales se adjunta su currículum vitae detallado.

El Dr. Marcelo Marciszack se desempeña actualmente como Director del Centro de Investigación, Desarrollo y Transferencia de Sistemas de Información (CIDS) en la Universidad Tecnológica Nacional, Facultad Regional Córdoba (Argentina), se encuentra categorizado A en la Carrera de Docente Investigador de la UTN -
Orientación Ciencias de la Ingeniería y Tecnológicas, y Categorizado I
en Programa de Incentivos del Ministerio de Ciencia y Tecnología.

El Dr. Carlos Feinstein es Dr. en Astronomía, por la Faculta de Ciencias Astronómicas y Geofísicas de la Universidad Nacional de La Plata (Argentina), es profesor Titular concursado en la Cátedra de Computación de la Facultad de Ciencias Astronómicas y Geofísicas, de la Universidad Nacional de La Plata, además de Investigador Independiente de CONICET desde noviembre de 2010 hasta la fecha, y está incluído en la Categoría II del Programa de Incentivos del Ministerio de Ciencia y Tecnología. 

\section {Infraestructura y equipamiento}

Las tareas se desarrollarán en las instalaciones del Centro de Investigación, Desarrollo y Transferencia de Sistemas de Información, en la Universidad Tecnológica Nacional, Facultad Regional Córdoba.

Dicha locación cuenta con los requerimientos necesarios para el desarrollo del trabajo en cuanto a equipamiento informático, conectividad, acceso a bases de datos, información de referencia, etc.

%\section {Bibliografía y material de referencia}

% Material CITADO, entra bajo el titulo REFERENCIAS
\printbibliography[category=cited,title={Referencias}]

% Material NO CITADO, entra bajo el titulo BIBLIOGRAFIA ADICIONAL
\defbibnote{bibnote}{El presente material bibliográfico se ha utilizado como material de estudio pero no se ha citado directamente en el texto.}
\printbibliography[title={Bibliografía adicional},prenote=bibnote,notcategory=cited]

\end{document}