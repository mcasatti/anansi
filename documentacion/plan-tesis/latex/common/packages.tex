% ======================================================
% PAQUETES DE GESTION DE IDIOMA Y CORRECCION
% ======================================================
\usepackage[utf8]{inputenc}
\usepackage[T1]{fontenc}
\usepackage[spanish]{babel}

% ======================================================
% MEJORA LA LEGIBILIDAD GENERAL DEL DOCUMENTO Y LAYOUT
% ======================================================
% Microtype tiene que preceder a Ragged2e
\usepackage[babel=true,expansion=true,protrusion=true]{microtype}
% Ragged2e tiene que ir despues de microtype para funcionar correctamente sino cambia las alineaciones y no quedan bien
\usepackage{ragged2e}
% Permite utilizar encabezados y pies de pagina simples con KOMA-Script
\usepackage{scrlayer-scrpage}
% Permite imprimir el layout de un documento, con toda la informacion importante
% mediante el comando
% \layout (en document)
% o
% \currentpage \pagedesign
% \currentpage \pagediagram \pagevalues
\usepackage{layouts}
\usepackage{float}
\usepackage{wrapfig}
\usepackage{caption}
% Para poder tachar texto con \uwave{}, uwave{}, sout{}
\usepackage[normalem]{ulem} % El parametro normalem es para no alterar el emphasis (italica)

% ======================================================
% GESTION DE COLORES Y GRAFICOS
% ======================================================
\usepackage[dvipsnames,svgnames]{xcolor}
\usepackage{graphicx}

% ======================================================
% GRAFICOS, CHARTS y CUADROS
% ======================================================
\usepackage{tkz-graph}
\usepackage{tikz}
\usepackage{pgfplots}
\usetikzlibrary{
	arrows,
	decorations.pathmorphing,
	backgrounds,
	positioning,
	calc,
	fit,
	petri,
	quotes,
	% Este paquete produce errores usado en conjunto con tkz-graph
	% babel,	
	arrows.meta,
	decorations.pathreplacing,
	shapes,
	datavisualization,
	plotmarks
}
\usepackage[framemethod=tikz]{mdframed}
\usepackage[most]{tcolorbox}

% ======================================================
% TODOLIST
% ======================================================
\usepackage[colorinlistoftodos,textsize=tiny]{todonotes}
\usepackage{xargs}

\newcommandx{\unsure}[2][1=]{\todo[linecolor=red,backgroundcolor=red!25,bordercolor=red,#1]{#2}}
\newcommandx{\change}[2][1=]{\todo[linecolor=blue,backgroundcolor=blue!25,bordercolor=blue,#1]{#2}}
\newcommandx{\info}[2][1=]{\todo[linecolor=green,backgroundcolor=green!25,bordercolor=green,#1]{#2}}
\newcommandx{\improvement}[2][1=]{\todo[linecolor=plum,backgroundcolor=plum!25,bordercolor=plum,#1]{#2}}
\newcommandx{\thiswillnotshow}[2][1=]{\todo[disable,#1]{#2}}

\newcommand{\revisar}[1]{\todo[backgroundcolor=yellow!50]{\textbf{Revisar:} #1}}
\newcommand{\ampliar}{\todo[backgroundcolor=blue!30]{Ampliar}}
\newcommand{\revisarcita}{\colorbox{red}{\textcolor{white}{\ CITA\ }}}
\newcommand{\sincita}{\colorbox{red}{\textcolor{white}{\ CITA\ }}}
\newcommand{\faltafigura}{\missingfigure{Falta figura\ldots}}
\newcommand{\citaurl}{\todo[backgroundcolor=green!50]{URL}}

% ======================================================
% MARCAS DE AGUA
% ======================================================
\usepackage{draftwatermark}

% ======================================================
% GANTT
% ======================================================
\usepackage{pgfgantt}

\usepackage{tablefootnote}
\usepackage{subcaption}
\usepackage[export]{adjustbox}
\usepackage[useregional]{datetime2}
\usepackage{datetime2-calc}
\usepackage{array}
\usepackage[hidelinks]{hyperref}
\usepackage{xstring}
\captionsetup[table]{belowskip=2pt}