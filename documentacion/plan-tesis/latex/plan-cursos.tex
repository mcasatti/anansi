\documentclass[
	11pt,oneside,a4paper,
	headsepline,footsepline,
	%plainfootsepline,plainheadsepline,
	fleqn,
	%flushbottom,
	%raggedbottom
]{memoir}
%]{scrartcl}
%]{article}
%\usepackage[a4paper, total={6in, 8in}]{geometry}

%------------------------------------------
% FLAG DE BORRADOR
%------------------------------------------
%\def\borrador{}
%------------------------------------------

%------------------------------------------
% CONFIGURACION DEL DOCUMENTO
%------------------------------------------

\setlength {\marginparwidth }{2cm}

%------------------------------------------
% INCLUSION DE PAQUETES COMUNES
%------------------------------------------
% Required for specifying colors by name
\usepackage[usenames,dvipsnames,svgnames,table]{xcolor} 
% Define the orange color used for highlighting throughout the book
\definecolor{ocre}{RGB}{243,102,25} 
\usepackage[export]{adjustbox}
\usepackage{subcaption}
\usepackage{float}
% Permite utilizar encabezados y pies de pagina simples con KOMA-Script
\usepackage{scrlayer-scrpage}
% Page margins
\usepackage[top=3cm,bottom=3cm,left=3cm,right=3cm,headsep=10pt,a4paper]{geometry} 
\usepackage[document]{ragged2e}
% Required for including pictures
\usepackage{graphicx} 
% Inserts dummy text
\usepackage{lipsum} 
%\usepackage[english]{babel} % English language/hyphenation
\usepackage[spanish,es-noquoting]{babel} % English language/hyphenation
% Customize lists
\usepackage{enumitem} 
% Resaltado de sintaxis en listados de código fuente
\usepackage{listings}
% Required for nicer horizontal rules in tables
\usepackage{booktabs} 
% Mejor manejo de tablas
\usepackage{tabularx} 
% Manejo de hipervinculos
\usepackage[hidelinks]{hyperref}
%Manejo de referencias y bibliografía
%\usepackage
%[
%	%style=alphabetic,
%	style=numeric,
%	citestyle=numeric,
%	sorting=none,
%	%sortcites=true,
%	%autopunct=true,
%	babel=hyphen,
%	hyperref=true,
%	abbreviate=false,
%	backref=true,
%	%backend=biblatex,
%	defernumbers=false
%]
%{biblatex}
% Permite sacar parte del codigo a un archivo externo. Util para referenciar bibliografía embebida
%\usepackage{filecontents}
%----------------------------------------------------------------------------------------
%	ALGORITMOS
%----------------------------------------------------------------------------------------
\usepackage{algorithm2e}
% Slightly tweak font spacing for aesthetics
%\usepackage{microtype} 
% Required for including letters with accents
\usepackage[utf8]{inputenc} 
% Use 8-bit encoding that has 256 glyphs
\usepackage[T1]{fontenc} 
% Agregar fuentes monoespaciadas bold
\usepackage{bold-extra}
% Using Courier font
%\renewcommand{\ttdefault}{pcr}
%----------------------------------------------------------------------------------------
%	DEFINICION DE DIVERSOS TIPOS DE TO-DO
%----------------------------------------------------------------------------------------
\usepackage[colorinlistoftodos,prependcaption,textsize=tiny]{todonotes}
\usepackage{tablefootnote}
\usepackage{xargs}

\newcommandx{\revisar}[2][1=]{\todo[linecolor=red,backgroundcolor=red!25,bordercolor=red,#1]{#2}}
\newcommandx{\cambiar}[2][1=]{\todo[linecolor=blue,backgroundcolor=blue!25,bordercolor=blue,#1]{#2}}
\newcommandx{\info}[2][1=]{\todo[linecolor=green,backgroundcolor=green!25,bordercolor=green,#1]{#2}}
\newcommandx{\mejorar}[2][1=]{\todo[linecolor=gray,backgroundcolor=Plum!25,bordercolor=gray,#1]{#2}}
\newcommandx{\faltafigura}[2][1=]{\missingfigure[#1,figcolor=white]{#2}}
\newcommandx{\nomostrar}[2][1=]{\todo[disable,#1]{#2}}

% Package para creacion de grafos simplificados
\usepackage{tkz-graph}
% Required for drawing custom shapes
\usepackage{tikz} 
\usetikzlibrary{
	arrows,
	decorations.pathmorphing,
	backgrounds,
	positioning,
	fit,
	petri,
	quotes,
	% Este paquete produce errores usado en conjunto con tkz-graph
	%babel,	
	arrows.meta,
	decorations.pathreplacing,
	shapes
}

%----------------------------------------------------------------------------------------
%	CONFIGURACION DE LOS LISTADOS DE CODIGO FUENTE
%----------------------------------------------------------------------------------------
\definecolor{armygreen}{rgb}{0.29, 0.33, 0.13}
\lstset{language=Java,
	tabsize=3,
	showstringspaces=false,
	numbers=left,
	basicstyle=\ttfamily,
	keywordstyle=\color{black}\ttfamily\bfseries,
	stringstyle=\color{armygreen}\ttfamily,
	commentstyle=\color{gray}\ttfamily,
	morecomment=[l][\color{gray}]{\#}
}

\usepackage{csquotes}
\usepackage[]{enumitem}
%------------------------------------------
% CONFIGURACION DE BIBLIOGRAFIA
%------------------------------------------
\usepackage[
	backend=biber,
	%sorting=anyt,  %% Alfabetico
	sorting=none,	%% En orden de aparición
	style=nature,
	%style=numeric,
	%style=unsrt
	%citestyle=numeric
]{biblatex}
\addbibresource{refs_used.bib}
%\nocite{*}

%\DeclareBibliographyCategory{cited}
%\AtEveryCitekey{\addtocategory{cited}{\thefield{entrykey}}}
%\addbibresource{refs_used.bib}
%\BiblatexSplitbibDefernumbersWarningOff

\urlstyle{sf}




\ifdefined\borrador
	\SetWatermarkText{Borrador}
	\SetWatermarkScale{0.8}
	\SetWatermarkLightness{0.9}
\else
	\SetWatermarkText{}
\fi

%------------------------------------------
% FUENTES
%------------------------------------------
%\usepackage[sfdefault,scaled=1]{FiraSans}
\usepackage[libertine,scaled=1]{newtx}
%------------------------------------------

\graphicspath{
	{figures/}
	{images}
}

\clearpairofpagestyles
%\ihead{\today}								% Header interno: Fecha
% ENCABEZADO
\ohead{\textit{Plan de Tesis}}		% Header interno: Titulo
\ihead{\raisebox{-7mm}{\includegraphics[height=10mm]{./images/utn-black}}}		% Footer interno: 
% PIE DE PAGINA
\ofoot{Página \pagemark}	% Header externo: Evento
%Titulo
%\ohead{\raisebox{-10mm}{\includegraphics[height=10mm]{./images/MISI3}}}

% Impedir que se comience a numerar a partir de capitulos (como el plan no tiene capitulos los numeros saldrían como 0.1, 0.2, etc)
\counterwithout{section}{chapter}

% INTERLINEADO 1 1/2 
%\OnehalfSpacing

\pgfplotsset{compat=1.18}
\renewcommand\fbox{\fcolorbox{gray!50}{white}}

\usepackage{pdflscape}

\begin{document}

\thispagestyle{empty}

\begin{center}
	\vspace*{3cm}
	
	\HUGE{Caracterización estructural de formaciones astronómicas, utilizando un enfoque de grafos, para reconocimiento de cumulos estelares en galaxias cercanas}
	
	\vspace{2cm}
	
	\LARGE{Universidad Tecnológica Nacional\\
		Facultad Regional Córdoba}
	
	\vspace{2cm}
	
	\LARGE{Proyecto de Investigación y Desarrollo (PID)}
	
	\vspace{2cm}
\end{center}

\begin{flushright} 
	\Large{
		Tesista: Esp. Ing. Martin Casatti\\
		Director: Dr. Oscar Medina\\
		CoDirectors: Mgr. Cynthia Corso
	}
\end{flushright} 


\newpage

\ifdefined\borrador
	\listoftodos
\fi 

\section*{Plan de cursos}

Se detalla en la siguiente sección el plan de cursos a realizar y cursos ya realizados (aquellos que tienen nota y fecha de aprobación).

El tipo de curso se define según la siguientes siglas:
\begin{description}
    \item[PF] PROFORVIN
    \item[PG] POSGRADO
    \item[MI] MAESTRÍA EN INGENIERÍA  
\end{description}

\begin{table}[H]
    \centering
    \begin{tabular}{|l|c|l|c|c|}
    \hline
    \textbf{Nombre} &
      \textbf{Carga horaria} &
      \textbf{Aprobado} &
      \textbf{Tipo} &
      \textbf{Nota} \\ \hline
    \begin{tabular}[c]{@{}l@{}}Introducción a la Investigación, \\ el Desarrollo y la Innovación.\end{tabular}                & 64 & 29/06/2017 & PF & 9                     \\ \hline
    \begin{tabular}[c]{@{}l@{}}Redes Neuronales y Lógica Difusa \\ en Ingeniería\end{tabular} &
      60 &
      05/03/2021 &
      PG &
      10 \\ \hline
    Introducción a la Ciencia de Datos &
      60 &
      02/10/2020 &
      PG &
      10 \\ \hline
    \begin{tabular}[c]{@{}l@{}}Métodos Empíricos en Ingeniería \\ de Software\end{tabular} &
      60 &
      2018/2019 &
      PG &
       \\ \hline
    \begin{tabular}[c]{@{}l@{}}Modelos de Organizaciones y Sistemas \\ de Información\end{tabular} &
      60 &
      19/05/2016 &
      MI &
      8 \\ \hline
    Ingeniería de Software(Elect.) &
      60 &
      29/07/2016 &
      MI &
      9 \\ \hline
    \begin{tabular}[c]{@{}l@{}}Modelado Conceptual de Sistemas \\ de Información(Elec.)\end{tabular} &
      60 &
      31/10/2016 &
      MI &
      9 \\ \hline
    \begin{tabular}[c]{@{}l@{}}C.A.P. Estimaciones de Software de \\ Sistemas de Información\end{tabular} &
      60 &
      14/11/2016 &
      MI &
      10 \\ \hline
    Seminario Integrador &
       &
      18/05/2018 &
      MI &
      10 \\ \hline
    Trabajo Final Integrador &
       &
      18/05/2018 &
      MI &
      10 \\ \hline
    \begin{tabular}[c]{@{}l@{}}Tópicos de Machine Learning aplicados\\ a Big Data\end{tabular} &
      36 &
       &
      PG &
      \multicolumn{1}{l|}{} \\ \hline
    \begin{tabular}[c]{@{}l@{}}Aspectos teóricos y metodológicos del \\ análisis computacional de redes sociales\end{tabular} & 60 &            & PG & \multicolumn{1}{l|}{} \\ \hline
    \end{tabular}
    \end{table}


\end{document}