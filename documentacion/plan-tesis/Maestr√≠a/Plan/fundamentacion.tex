	\ifdefined\borrador
	\info[inline]{Información de contexto sobre el ámbito en el que se va a desarrollar el trabajo}
	\fi
	
	\unsure[inline]{GIRÓ:Bien pero ampliar. No hay referencias a antecedentes o motivación del tesista por el tema abordado}
	
	El problema de la representación del conocimiento de tal forma que sea procesable por técnicas automáticas no es nuevo y abarca prácticamente toda la historia de la informática como disciplina. Todos los mecanismos de almacenamiento de información tienen como finalidad última el permitir que un sistema de cómputo pueda acceder y procesar dicha información sin la intervención humana.
	
	Las bases de datos relacionales son quizá el mecanismo más establecido y popular a la hora de almacenar información que debe ser procesada por una computadora. Pero existe un problema inherente a la representación y consulta de la información almacenada y es que para que las consultas puedan realizarse en lenguaje natural\footnote{En la filosofía del lenguaje, el lenguaje natural es la lengua o idioma hablado o escrito por humanos para propósitos generales de comunicación. Se diferencia de los lenguajes artificiales, que han sido diseñados con finalidades específicas.} el usuario no tiene por qué conocer la estructura en la cual se hayan almacenados los datos.
	
	Como vemos, las bases de datos relacionales no cumplen con este requisito ya que su estructura, modelada como tablas, compuestas cada una por filas y columnas, debe ser perfectamente conocida para permitir la consulta y el acceso a la información representada.
	
	En 1976, un trabajo de John Sowa, un investigador de IBM (International Bussiness Machines), define las bases de lo que se conoce como grafos conceptuales, con el objetivo de poder realizar consultas a una base de datos sin conocer la estructura subyacente de la misma.\cite{Sowa1976}
	
	La representación del conocimiento es una rama de la Inteligencia Artificial (IA) que se dedica a estudiar la representación del conocimiento del mundo de forma tal que una computadora pueda interpretarla y utilizarla para resolver problemas complejos. Y es dentro de este ámbito que los grafos conceptuales se han probado de gran valor.
	
	En palabras de Marko Rodriguez, Doctor por la Universidad de California en Santa Cruz:
	
	\begin{center}
		\begin{minipage}{0.9\linewidth}
			\vspace{5pt}%margen superior de minipage
			{\small
				``Una base de datos de grafos, y su ecosistema asociado, puede llevarnos a una solución elegante y eficiente a diversos problemas en el ámbito de la representación del conocimiento y el razonamiento.''
			}
			\begin{flushright}
				(\citeauthor{markorodriguez2011}, \citeyear{markorodriguez2011})
			\end{flushright}
			\vspace{5pt}%margen inferior de la minipage
		\end{minipage}
	\end{center}
	
	En el año 1983, John Sowa ampliaría y compaginaría todos los trabajos previos realizados sobre sus ideas de los grafos conceptuales en el libro \citetitle{Sowa1983}, en donde tocaría temas tan diversos como la filosofía, la psicología, la lingüística y la inteligencia artificial.
	
	Actualmente se deben tener en cuenta al menos cuatro factores fundamentales a la hora de diseñar un sistema de representación del conocimiento en cualquier dominio dado:
	
	\begin{description}
		\item [Adecuación Representacional:] Habilidad para representar todas las clases de conocimiento que son necesarias en el dominio.\cite{van2008handbook}
		\item [Adecuación Inferencial:] Habilidad de manipular estructuras de representación de tal manera que devengan o generen nuevas estructuras que correspondan a nuevos conocimientos inferidos de los anteriores.\cite{van2008handbook}
		\item [Eficiencia Inferencial:] Capacidad del sistema para incorporar información adicional a la estructura de representación, llamada metaconocimiento, que puede emplearse para focalizar la atención de los mecanismos de inferencia con el fin de optimizar los cómputos.\cite{van2008handbook}
		\item [Eficiencia en la Adquisición:] Capacidad de incorporar fácilmente nueva información. Idealmente el sistema por sí mismo deberá ser capaz de controlar la adquisición de nueva información y su posterior representación.\cite{van2008handbook}
	\end{description}
