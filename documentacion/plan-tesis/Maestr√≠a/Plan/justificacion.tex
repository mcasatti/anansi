	\ifdefined\borrador
	\info[inline]{Por qué se eligió desarrollar el trabajo}
	\fi
	
	Actualmente la mayor dificultad que pueden encontrar las personas no es el acceso a la información en si sino la selección de información relevante frente a un volumen de datos que crece constantemente.
	
	Con una cantidad de datos en constante aumento y la multiplicación de las diversas fuentes que proveen esta información, el volumen de contenido a analizar aumenta exponencialmente y son cada vez más necesarios métodos automatizados para su proceso y categorización.
	
	Por otra parte, tan importantes como los datos en sí, resultan ser las relaciones entre los mismos, lo que establece un conjunto completamente nuevo de temas a considerar, tales como la cardinalidad y la direccionalidad de las relaciones, la distancia entre datos y medidas tales como la centralidad, el contorno y el peso de los componentes, entre otras. 
	
	Uno de los campos que puede obtener claros beneficios al aplicar técnicas de reconocimiento de patrones sobre grandes conjuntos de datos es la astronomía, especialmente en la rama que estudia las agrupaciones estelares.

	Se dispone actualmente de una gran cantidad de información de galaxias cercanas, obtenidos a traves de observaciones realizadas por medio del Telescopio Espacial Hubble (HST\footnote{Hubble Space Telescope}), consiguiendo imágenes de alta resolución por medio de diversas cámaras de campo amplio (WFPC2, ACS\cite{dalcanton2009acs}).
	
	Se consigue, a través de las mismas, realizar fotometría multicolor precisa para estudiar estrellas individuales y estructuras en las escala de unos pocos parsecs\footnote{Unidad de medida en astronomía equivalente a aproximadamente 3.2616 años luz} en dichas galaxias.
	
	En la actualidad existe una gran cantidad de información de las galaxias cercanas (a varios Mpc) debido, en gran parte, a que el Telescopio Espacial Hubble (HST) ha permitido obtener datos con alta resolución espacial utilizando varias cámaras de campo amplio (WFPC2; ACS; ). De esta forma, es posible hacer fotometría precisa multicolor para estudiar estrellas individuales y estructuras en la escala de unos pocos parsecs en dichas galaxias. Consecuentemente, se pueden conducir investigaciones vinculadas con agrupaciones estelares, diferentes poblaciones estelares e historias de formación estelar en ambientes muy diferentes al de la Vía Láctea.

Existe una enorme cantidad de datos proveniente de las varias observaciones continuas que se están realizando y que se proyectan realizar en modo “survey” (p.e. VVV, https://vvvsurvey.org/ o LSST; https://www.lsst.org/)\revisarcita que necesitan ser estudiados con métodos automáticos. En particular para la identificación y parametrización de nuevas agrupaciones estelares

\cite{schmeja2011identifying}

	\opensacar
	La elaboración de un modelo para la representación de datos heterogéneos de forma consistente y el desarrollo de técnicas de consulta tendientes a obtener información valiosa, no de uno o más datos puntuales, sino desde el punto de vista de estructuras de información completas, reviste fundamental importancia y tiene el potencial de brindar la base para el desarrollo de herramientas nuevas y poderosas para obtener información valiosa a partir de una cantidad enorme de datos relacionados, con independencia de la fuente que los origina.
	
	El presente trabajo propone un plan de tesis que buscar abordar el problema de representación de información a partir de fuentes heterogéneas, de una manera completa y consistente que posibilite usar técnicas de consulta para la detección de patrones entre los datos.
	\closesacar
	
	
	\revisar{Poner algo de información de qué son las agrupaciones estelares, citar trabajos de Carlos Feinstein al respecto}
	\ampliar \revisarcita
