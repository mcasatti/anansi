	\unsure[inline]{GIRÓ: Muy orientado a detallar la secuencia de actividades y muy escaso el contenido metodológico: es decir, indicar ``cómo'' voy a hacer las cosas, no el ``qué'' voy a hacer}
	
	\begin{center}
		\begin{minipage}{0.9\linewidth}
			\vspace{5pt}%margen superior de minipage
			\subsection{Hipótesis}
			``El uso de grafos conceptuales para la representación de información heterogénea, permitirá la detección de patrones de comportamiento asociados al tráfico de mercadería en una frontera geográfica.''
			\vspace{5pt}%margen inferior de la minipage
		\end{minipage}
	\end{center}
	
	
	En el presente trabajo se iniciará evaluando las distintas alternativas disponibles para el almacenamiento de información en forma de grafos dirigidos, para soportar el modelo de grafos conceptuales de \citeauthor{Sowa1976}.
	
	A continuación se realizará un análisis detallado de las distintas fuentes de información actualmente disponibles para poder determinar un conjunto de atributos que permitan representar cada una de ellas. Estos atributos o descriptores se consolidarán luego de forma tal que se definan un conjunto de descriptores comunes a todas las fuentes de información y un conjunto propio de cada una de las fuentes.\cite{findler2014associative}
	
	Una vez determinados los descriptores generales y particulares se modelarán los distintos tipos de relaciones que pueden unir dos conceptos cualquiera y se definirán los atributos asociados a las mismas.
	
	Una vez que los modelos conceptuales estén plasmados en la base de datos elegida, se diseñaran los algoritmos necesarios que permitirán adquirir la información de las distintas fuentes y que, luego de realizar los mapeos y adecuaciones necesarios, registraran la misma en la base de conocimientos, con el mayor nivel de detalle posible, y establecerán las relaciones necesarias con la información asociada, ya existente en la base de conocimentos.
	
	Habiendo concluido con el modelado conceptual de la base de conocimiento y teniendo implementados los mecanismos de carga inicial y mantenimiento, se desarrollarán algoritmos de búsqueda que tengan en cuenta el modelo conceptual existente en la base de datos. Hay que mencionar que no se realizará una consulta relacional sino conceptual, de más alto nivel, pero sin llegar a implementar una consulta en lenguaje natural.
	
	Se implementarán, posteriormente, el análisis de métricas del grafo conceptual constituído, lo cual dará lugar a la detección de patrones que no son evidentes o detectables mediante la simple consulta de los datos. Se buscarán medidas de centralidad, dispersión, cercanía, camino más corto, más largo, perímetro y otras que sirvan para caracteriza la información del grafo o un subconjunto de la misma.
	
	Finalmente se elaborarán las conclusiones, necesarias para determinar si la utilización de un grafo conceptual es un mecanismo válido y eficiente para la representación de información heterogénea para detección de patrones, y se enunciarán los trabajos futuros que surjan a partir de dicho análisis.
