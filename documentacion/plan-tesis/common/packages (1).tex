%
% PAQUETES DE GESTION DE IDIOMA Y CORRECCION
%
\usepackage[utf8]{inputenc}
\usepackage[T1]{fontenc}
% 
% MEJORA LA LEGIBILIDAD GENERAL DEL DOCUMENTO
%
\usepackage[spanish]{babel}
\usepackage{layouts}
% Microtype tiene que preceder a Ragged2e
\usepackage[babel=true,expansion=true,protrusion=true]{microtype}
% Ragged2e tiene que ir despues de microtype para funcionar correctamente sino cambia las alineaciones y no quedan bien
\usepackage{ragged2e}
%
% GESTION DE COLORES Y GRAFICOS
%
\usepackage[dvipsnames,svgnames]{xcolor}
\usepackage{graphicx}
\usepackage{tkz-graph}
\usepackage{tikz}
\usepackage{pgfplots}
\usetikzlibrary{
	arrows,
	decorations.pathmorphing,
	backgrounds,
	positioning,
	calc,
	fit,
	petri,
	quotes,
	% Este paquete produce errores usado en conjunto con tkz-graph
	% babel,	
	arrows.meta,
	decorations.pathreplacing,
	shapes,
	datavisualization,
	plotmarks
}

\usepackage{array}
\usepackage[hidelinks]{hyperref}
\usepackage{xstring}
\usepackage{xargs}
\usepackage{float}
\usepackage{wrapfig}
\usepackage{caption}
\usepackage[framemethod=tikz]{mdframed}

% TODOLIST
\usepackage[colorinlistoftodos,textsize=tiny]{todonotes}
% Para poder tachar texto con \uwave{}, uwave{}, sout{}
\usepackage[normalem]{ulem} % El parametro normalem es para no alterar el emphasis (italica)
% ORNAMENTOS GRAFICOS
\usepackage[object=vectorian]{pgfornament}
% LETRA CAPITAL EN EL TEXTO
\usepackage{lettrine}
% MARCAS DE AGUA
\usepackage{draftwatermark}
% GANTT
\usepackage{pgfgantt}

\usepackage{tablefootnote}
\usepackage{subcaption}
\usepackage[export]{adjustbox}
\usepackage[useregional]{datetime2}
\usepackage{datetime2-calc}
% Permite utilizar encabezados y pies de pagina simples con KOMA-Script
\usepackage{scrlayer-scrpage}
% Permite imprimir el layout de un documento, con toda la informacion importante
% mediante el comando
% \layout (en document)
% o
% \currentpage \pagedesign
% \currentpage \pagediagram \pagevalues
\usepackage{layouts}

\renewcommand\epigraphrule{0pt}
\setlength\epigraphwidth{.8\textwidth}
\newcommand{\dictum}[3]{\epigraph{\textit{#1}}{--- #2, \textit{#3}}}
\newcommand{\epigrafe}[1]{\epigraph{\textit{#1}}}
\renewcommand{\quote}[1]{
	\begin{quotation}
		\emph{``#1''}
	\end{quotation}}

\newcommand{\firstword}[1]{
	\lettrine[lines=2]{
		\color{NavyBlue}\StrLeft{#1}{1}
	}{
		\StrGobbleLeft{#1}{1}
	}
}

%\lettrine[lines=4]{\color{BrickRed}S}{tart} of the c
\newcommand{\separador}{
	\begin{center}
		\bigskip
		\pgfornament[anchor=center,width=10cm,color=gray]{88}
	\end{center}
}

\newcommand{\hl}[1]{
	\begin{mdframed}[
		%hidealllines=true,
		backgroundcolor=yellow!30,
		innerleftmargin=10pt,
		innerrightmargin=10pt,
		leftmargin=-3pt,
		rightmargin=-3pt
		]
		#1
	\end{mdframed}
}
\newcommand{\hlr}[1]{
	\begin{mdframed}[
		%hidealllines=true,
		backgroundcolor=red!30,
		innerleftmargin=10pt,
		innerrightmargin=10pt,
		leftmargin=-3pt,
		rightmargin=-3pt
		]
		#1
	\end{mdframed}
}

\newcommand{\opennuevo}{\hl{\centering NUEVO $\Downarrow$}}
\newcommand{\closenuevo}{\hl{\centering NUEVO $\Uparrow$}}
\newcommand{\opensacar}{\hlr{\centering SACAR $\Downarrow$}}
\newcommand{\closesacar}{\hlr{\centering SACAR $\Uparrow$}}

\newcommandx{\unsure}[2][1=]{\todo[linecolor=red,backgroundcolor=red!25,bordercolor=red,#1]{#2}}
\newcommandx{\change}[2][1=]{\todo[linecolor=blue,backgroundcolor=blue!25,bordercolor=blue,#1]{#2}}
\newcommandx{\info}[2][1=]{\todo[linecolor=green,backgroundcolor=green!25,bordercolor=green,#1]{#2}}
\newcommandx{\improvement}[2][1=]{\todo[linecolor=plum,backgroundcolor=plum!25,bordercolor=plum,#1]{#2}}
\newcommandx{\thiswillnotshow}[2][1=]{\todo[disable,#1]{#2}}

\newcommand{\revisar}[1]{\todo[backgroundcolor=yellow!50,inline]{#1}}
\newcommand{\ampliar}{\todo[backgroundcolor=blue!30,inline]{Ampliar}}
\newcommand{\revisarcita}{\colorbox{red}{\textcolor{white}{\ CITA\ }}}
\newcommand{\faltafigura}{\missingfigure{Falta figura\ldots}}
\newcommand{\citaurl}{\todo[backgroundcolor=green!50]{URL}}

\captionsetup[table]{belowskip=2pt}
