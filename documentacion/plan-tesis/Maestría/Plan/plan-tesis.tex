\documentclass[
	11pt,oneside,a4paper,
	headsepline,footsepline,plainfootsepline,plainheadsepline,
	fleqn,
	flushbottom,
	raggedbottom
]{memoir}
%]{scrartcl}
	
%\usepackage[inner=25mm, outer=25mm, top=25mm, bottom=20mm]{geometry} 

\def\borrador{}
\counterwithout{section}{chapter}

% Required for specifying colors by name
\usepackage[usenames,dvipsnames,svgnames,table]{xcolor} 
% Define the orange color used for highlighting throughout the book
\definecolor{ocre}{RGB}{243,102,25} 
\usepackage[export]{adjustbox}
\usepackage{subcaption}
\usepackage{float}
% Permite utilizar encabezados y pies de pagina simples con KOMA-Script
\usepackage{scrlayer-scrpage}
% Page margins
\usepackage[top=3cm,bottom=3cm,left=3cm,right=3cm,headsep=10pt,a4paper]{geometry} 
\usepackage[document]{ragged2e}
% Required for including pictures
\usepackage{graphicx} 
% Inserts dummy text
\usepackage{lipsum} 
%\usepackage[english]{babel} % English language/hyphenation
\usepackage[spanish,es-noquoting]{babel} % English language/hyphenation
% Customize lists
\usepackage{enumitem} 
% Resaltado de sintaxis en listados de código fuente
\usepackage{listings}
% Required for nicer horizontal rules in tables
\usepackage{booktabs} 
% Mejor manejo de tablas
\usepackage{tabularx} 
% Manejo de hipervinculos
\usepackage[hidelinks]{hyperref}
%Manejo de referencias y bibliografía
%\usepackage
%[
%	%style=alphabetic,
%	style=numeric,
%	citestyle=numeric,
%	sorting=none,
%	%sortcites=true,
%	%autopunct=true,
%	babel=hyphen,
%	hyperref=true,
%	abbreviate=false,
%	backref=true,
%	%backend=biblatex,
%	defernumbers=false
%]
%{biblatex}
% Permite sacar parte del codigo a un archivo externo. Util para referenciar bibliografía embebida
%\usepackage{filecontents}
%----------------------------------------------------------------------------------------
%	ALGORITMOS
%----------------------------------------------------------------------------------------
\usepackage{algorithm2e}
% Slightly tweak font spacing for aesthetics
%\usepackage{microtype} 
% Required for including letters with accents
\usepackage[utf8]{inputenc} 
% Use 8-bit encoding that has 256 glyphs
\usepackage[T1]{fontenc} 
% Agregar fuentes monoespaciadas bold
\usepackage{bold-extra}
% Using Courier font
%\renewcommand{\ttdefault}{pcr}
%----------------------------------------------------------------------------------------
%	DEFINICION DE DIVERSOS TIPOS DE TO-DO
%----------------------------------------------------------------------------------------
\usepackage[colorinlistoftodos,prependcaption,textsize=tiny]{todonotes}
\usepackage{tablefootnote}
\usepackage{xargs}

\newcommandx{\revisar}[2][1=]{\todo[linecolor=red,backgroundcolor=red!25,bordercolor=red,#1]{#2}}
\newcommandx{\cambiar}[2][1=]{\todo[linecolor=blue,backgroundcolor=blue!25,bordercolor=blue,#1]{#2}}
\newcommandx{\info}[2][1=]{\todo[linecolor=green,backgroundcolor=green!25,bordercolor=green,#1]{#2}}
\newcommandx{\mejorar}[2][1=]{\todo[linecolor=gray,backgroundcolor=Plum!25,bordercolor=gray,#1]{#2}}
\newcommandx{\faltafigura}[2][1=]{\missingfigure[#1,figcolor=white]{#2}}
\newcommandx{\nomostrar}[2][1=]{\todo[disable,#1]{#2}}

% Package para creacion de grafos simplificados
\usepackage{tkz-graph}
% Required for drawing custom shapes
\usepackage{tikz} 
\usetikzlibrary{
	arrows,
	decorations.pathmorphing,
	backgrounds,
	positioning,
	fit,
	petri,
	quotes,
	% Este paquete produce errores usado en conjunto con tkz-graph
	%babel,	
	arrows.meta,
	decorations.pathreplacing,
	shapes
}

%----------------------------------------------------------------------------------------
%	CONFIGURACION DE LOS LISTADOS DE CODIGO FUENTE
%----------------------------------------------------------------------------------------
\definecolor{armygreen}{rgb}{0.29, 0.33, 0.13}
\lstset{language=Java,
	tabsize=3,
	showstringspaces=false,
	numbers=left,
	basicstyle=\ttfamily,
	keywordstyle=\color{black}\ttfamily\bfseries,
	stringstyle=\color{armygreen}\ttfamily,
	commentstyle=\color{gray}\ttfamily,
	morecomment=[l][\color{gray}]{\#}
}

\usepackage{setspace}

\ifdefined\borrador
	\SetWatermarkText{Borrador}
	\SetWatermarkScale{0.8}
\else
	\SetWatermarkText{}
\fi

\usepackage[sfdefault,scaled=1]{FiraSans}
%\usepackage{times}
%\usepackage{lmodern} % change by times,bookman,palatino, etc.

%\setcounter{tocdepth}{2}

% 
% LA GESTION BIBLIOGRAFICA NO ES COMUN A TODOS LOS DOCUMENTOS
% POR ESO LA DEFINO EN CADA UNO
%
%\usepackage[backend=biber,defernumbers=true,sorting=none]{biblatex}
\usepackage[
	backend=biber,
	%sorting=anyt,  %% Alfabetico
	sorting=none,	%% En orden de aparición
	style=numeric,
	citestyle=numeric
]{biblatex}
\DeclareBibliographyCategory{cited}
\AtEveryCitekey{\addtocategory{cited}{\thefield{entrykey}}}
%\addbibresource{bibliografia.bib}
\addbibresource{refs.bib}
\nocite{*}

\clearpairofpagestyles
%\ihead{\today}								% Header interno: Fecha
% ENCABEZADO
\ohead{\textit{Plan de Tesis de Maestría}}		% Header interno: Titulo
\ihead{\raisebox{-7mm}{\includegraphics[height=10mm]{./images/utn-black}}}		% Footer interno: 
% PIE DE PAGINA
\ofoot{Página \pagemark}	% Header externo: Evento
%Titulo
%\ohead{\raisebox{-10mm}{\includegraphics[height=10mm]{./images/MISI3}}}

%\title{Bases de datos de grafos como mecanismo de representación de información heterogénea para la detección de patrones}
%\date{\today}
%\author{Ing. Martin Casatti\\ UTN Regional Córdoba}

%\usepackage{type1cm}
%\usepackage{eso-pic}
%\makeatletter
%\AddToShipoutPicture{%
%	\setlength{\@tempdimb}{.5\paperwidth}%
%	\setlength{\@tempdimc}{.5\paperheight}%
%	\setlength{\unitlength}{1pt}%
%	\put(\strip@pt\@tempdimb,\strip@pt\@tempdimc){%
%		\makebox(0,0){\rotatebox{45}{\textcolor[gray]{0.75}%
%				{\fontsize{6cm}{6cm}\selectfont{DRAFT}}}}%
%	}%
%}
%\makeatother

%\onehalfspacing

\begin{document}
	
	%\currentpage
	%\pagedesign
	%\pagediagram
	%\pagevalues
	
	\thispagestyle{empty}
\begin{center}
	\begin{figure}[h]
		\begin{subfigure}{0.5\textwidth}
			\centering
			\includegraphics[width=5cm,left]{./images/utn-black}
		\end{subfigure}
		~
		\begin{subfigure}{0.5\textwidth}
			\centering
			\includegraphics[scale=0.8,right]{./images/MISI3}
		\end{subfigure}
	\end{figure}
	\vspace{1cm}
	\begin{minipage}{0.90\linewidth}
		\centering
		%University logo
		
		%\rule{0.4\linewidth}{0.15\linewidth}\par
		%\vspace{3cm}
		%Thesis title
		{\uppercase{\Large \textbf{Utilización de bases de datos de grafos para la detección de patrones en cúmulos estelares}\par}}
		\vspace{2cm}
		%Author's name
		{\Large Martín Gustavo Casatti\par}
		\vspace{2cm}
		%Degree
		{\Large Plan de Tesis para el Título de Magister en Ingeniería de Sistemas\par}
		\vspace{2cm}
		%Date
		{\Large \DTMMonthname{\the\month} de \the\year}
		\vspace{1cm}
	\end{minipage}
\end{center}
\noindent El presente trabajo se realiza como parte de los requerimientos para la obtención del título de Magíster en Ingeniería de Sistemas, tal como dicta la normativa de la Universidad Tecnológica Nacional.\\

\noindent \hfill\textit{\Today} \\% Printing/edition date
\clearpage
	%\maketitle
	\tableofcontents % Print the table of contents itself

	\newpage
	
	\section{Justificación}
		\ifdefined\borrador
	\info[inline]{Por qué se eligió desarrollar el trabajo}
	\fi
	
	Actualmente la mayor dificultad que pueden encontrar las personas no es el acceso a la información en si sino la selección de información relevante frente a un volumen de datos que crece constantemente.
	
	Con una cantidad de datos en constante aumento y la multiplicación de las diversas fuentes que proveen esta información, el volumen de contenido a analizar aumenta exponencialmente y son cada vez más necesarios métodos automatizados para su proceso y categorización.
	
	Por otra parte, tan importantes como los datos en sí, resultan ser las relaciones entre los mismos, lo que establece un conjunto completamente nuevo de temas a considerar, tales como la cardinalidad y la direccionalidad de las relaciones, la distancia entre datos y medidas tales como la centralidad, el contorno y el peso de los componentes, entre otras. 
	
	Uno de los campos que puede obtener claros beneficios al aplicar técnicas de reconocimiento de patrones sobre grandes conjuntos de datos es la astronomía, especialmente en la rama que estudia las agrupaciones estelares.

	Se dispone actualmente de una gran cantidad de información de galaxias cercanas, obtenidos a traves de observaciones realizadas por medio del Telescopio Espacial Hubble (HST\footnote{Hubble Space Telescope}), consiguiendo imágenes de alta resolución por medio de diversas cámaras de campo amplio (WFPC2, ACS\cite{dalcanton2009acs}).
	
	Se consigue, a través de las mismas, realizar fotometría multicolor precisa para estudiar estrellas individuales y estructuras en las escala de unos pocos parsecs\footnote{Unidad de medida en astronomía equivalente a aproximadamente 3.2616 años luz} en dichas galaxias.
	
	En la actualidad existe una gran cantidad de información de las galaxias cercanas (a varios Mpc) debido, en gran parte, a que el Telescopio Espacial Hubble (HST) ha permitido obtener datos con alta resolución espacial utilizando varias cámaras de campo amplio (WFPC2; ACS; ). De esta forma, es posible hacer fotometría precisa multicolor para estudiar estrellas individuales y estructuras en la escala de unos pocos parsecs en dichas galaxias. Consecuentemente, se pueden conducir investigaciones vinculadas con agrupaciones estelares, diferentes poblaciones estelares e historias de formación estelar en ambientes muy diferentes al de la Vía Láctea.

Existe una enorme cantidad de datos proveniente de las varias observaciones continuas que se están realizando y que se proyectan realizar en modo “survey” (p.e. VVV, https://vvvsurvey.org/ o LSST; https://www.lsst.org/)\revisarcita que necesitan ser estudiados con métodos automáticos. En particular para la identificación y parametrización de nuevas agrupaciones estelares

\cite{schmeja2011identifying}

	\opensacar
	La elaboración de un modelo para la representación de datos heterogéneos de forma consistente y el desarrollo de técnicas de consulta tendientes a obtener información valiosa, no de uno o más datos puntuales, sino desde el punto de vista de estructuras de información completas, reviste fundamental importancia y tiene el potencial de brindar la base para el desarrollo de herramientas nuevas y poderosas para obtener información valiosa a partir de una cantidad enorme de datos relacionados, con independencia de la fuente que los origina.
	
	El presente trabajo propone un plan de tesis que buscar abordar el problema de representación de información a partir de fuentes heterogéneas, de una manera completa y consistente que posibilite usar técnicas de consulta para la detección de patrones entre los datos.
	\closesacar
	
	
	\revisar{Poner algo de información de qué son las agrupaciones estelares, citar trabajos de Carlos Feinstein al respecto}
	\ampliar \revisarcita

	
	
	\section{Fundamentación}
		\ifdefined\borrador
	\info[inline]{Información de contexto sobre el ámbito en el que se va a desarrollar el trabajo}
	\fi
	
	\unsure[inline]{GIRÓ:Bien pero ampliar. No hay referencias a antecedentes o motivación del tesista por el tema abordado}
	
	El problema de la representación del conocimiento de tal forma que sea procesable por técnicas automáticas no es nuevo y abarca prácticamente toda la historia de la informática como disciplina. Todos los mecanismos de almacenamiento de información tienen como finalidad última el permitir que un sistema de cómputo pueda acceder y procesar dicha información sin la intervención humana.
	
	Las bases de datos relacionales son quizá el mecanismo más establecido y popular a la hora de almacenar información que debe ser procesada por una computadora. Pero existe un problema inherente a la representación y consulta de la información almacenada y es que para que las consultas puedan realizarse en lenguaje natural\footnote{En la filosofía del lenguaje, el lenguaje natural es la lengua o idioma hablado o escrito por humanos para propósitos generales de comunicación. Se diferencia de los lenguajes artificiales, que han sido diseñados con finalidades específicas.} el usuario no tiene por qué conocer la estructura en la cual se hayan almacenados los datos.
	
	Como vemos, las bases de datos relacionales no cumplen con este requisito ya que su estructura, modelada como tablas, compuestas cada una por filas y columnas, debe ser perfectamente conocida para permitir la consulta y el acceso a la información representada.
	
	En 1976, un trabajo de John Sowa, un investigador de IBM (International Bussiness Machines), define las bases de lo que se conoce como grafos conceptuales, con el objetivo de poder realizar consultas a una base de datos sin conocer la estructura subyacente de la misma.\cite{Sowa1976}
	
	La representación del conocimiento es una rama de la Inteligencia Artificial (IA) que se dedica a estudiar la representación del conocimiento del mundo de forma tal que una computadora pueda interpretarla y utilizarla para resolver problemas complejos. Y es dentro de este ámbito que los grafos conceptuales se han probado de gran valor.
	
	En palabras de Marko Rodriguez, Doctor por la Universidad de California en Santa Cruz:
	
	\begin{center}
		\begin{minipage}{0.9\linewidth}
			\vspace{5pt}%margen superior de minipage
			{\small
				``Una base de datos de grafos, y su ecosistema asociado, puede llevarnos a una solución elegante y eficiente a diversos problemas en el ámbito de la representación del conocimiento y el razonamiento.''
			}
			\begin{flushright}
				(\citeauthor{markorodriguez2011}, \citeyear{markorodriguez2011})
			\end{flushright}
			\vspace{5pt}%margen inferior de la minipage
		\end{minipage}
	\end{center}
	
	En el año 1983, John Sowa ampliaría y compaginaría todos los trabajos previos realizados sobre sus ideas de los grafos conceptuales en el libro \citetitle{Sowa1983}, en donde tocaría temas tan diversos como la filosofía, la psicología, la lingüística y la inteligencia artificial.
	
	Actualmente se deben tener en cuenta al menos cuatro factores fundamentales a la hora de diseñar un sistema de representación del conocimiento en cualquier dominio dado:
	
	\begin{description}
		\item [Adecuación Representacional:] Habilidad para representar todas las clases de conocimiento que son necesarias en el dominio.\cite{van2008handbook}
		\item [Adecuación Inferencial:] Habilidad de manipular estructuras de representación de tal manera que devengan o generen nuevas estructuras que correspondan a nuevos conocimientos inferidos de los anteriores.\cite{van2008handbook}
		\item [Eficiencia Inferencial:] Capacidad del sistema para incorporar información adicional a la estructura de representación, llamada metaconocimiento, que puede emplearse para focalizar la atención de los mecanismos de inferencia con el fin de optimizar los cómputos.\cite{van2008handbook}
		\item [Eficiencia en la Adquisición:] Capacidad de incorporar fácilmente nueva información. Idealmente el sistema por sí mismo deberá ser capaz de controlar la adquisición de nueva información y su posterior representación.\cite{van2008handbook}
	\end{description}

	
	\section{Objetivos}
		\unsure[inline]{GIRÓ: Deben agruparse en principal y secundarios. No es conveniente que haya llamadas a pié de página. Cualquier aclaración de terminología debe hacer sido hecha con anterioridad}
	
	\begin{itemize}
		\item Definir las estructuras que permitan registrar y relacionar información no homogénea\footnote{Se considera información ``no homogénea'' a toda aquella información que difiera de otra tanto en la estructura de representación como en el tipo de datos que contiene.} que pueda ser utilizada para inferir patrones de comportamiento aplicables al tráfico de mercaderías a traves de una frontera.
		\begin{itemize}
			\item Definir los atributos generales y particulares que modelaran el tipo de contenidos y sus relaciones.
			\item Diseñar mecanismos de recolección automatizada de información que permitan la actualización permanente y automática de la base de conocimientos.
			\item Diseñar algoritmos que permitan la detección de patrones conocidos y la inferencia de patrones desconocidos en las relaciones entre los datos.
			\item Implementar, en la forma de un proyecto piloto, las herramientas diseñadas de forma tal que se demuestre la validez de las técnicas mencionadas.
		\end{itemize}
	\end{itemize}

	
	\section{Metodología}
		\unsure[inline]{GIRÓ: Muy orientado a detallar la secuencia de actividades y muy escaso el contenido metodológico: es decir, indicar ``cómo'' voy a hacer las cosas, no el ``qué'' voy a hacer}
	
	\begin{center}
		\begin{minipage}{0.9\linewidth}
			\vspace{5pt}%margen superior de minipage
			\subsection{Hipótesis}
			``El uso de grafos conceptuales para la representación de información heterogénea, permitirá la detección de patrones de comportamiento asociados al tráfico de mercadería en una frontera geográfica.''
			\vspace{5pt}%margen inferior de la minipage
		\end{minipage}
	\end{center}
	
	
	En el presente trabajo se iniciará evaluando las distintas alternativas disponibles para el almacenamiento de información en forma de grafos dirigidos, para soportar el modelo de grafos conceptuales de \citeauthor{Sowa1976}.
	
	A continuación se realizará un análisis detallado de las distintas fuentes de información actualmente disponibles para poder determinar un conjunto de atributos que permitan representar cada una de ellas. Estos atributos o descriptores se consolidarán luego de forma tal que se definan un conjunto de descriptores comunes a todas las fuentes de información y un conjunto propio de cada una de las fuentes.\cite{findler2014associative}
	
	Una vez determinados los descriptores generales y particulares se modelarán los distintos tipos de relaciones que pueden unir dos conceptos cualquiera y se definirán los atributos asociados a las mismas.
	
	Una vez que los modelos conceptuales estén plasmados en la base de datos elegida, se diseñaran los algoritmos necesarios que permitirán adquirir la información de las distintas fuentes y que, luego de realizar los mapeos y adecuaciones necesarios, registraran la misma en la base de conocimientos, con el mayor nivel de detalle posible, y establecerán las relaciones necesarias con la información asociada, ya existente en la base de conocimentos.
	
	Habiendo concluido con el modelado conceptual de la base de conocimiento y teniendo implementados los mecanismos de carga inicial y mantenimiento, se desarrollarán algoritmos de búsqueda que tengan en cuenta el modelo conceptual existente en la base de datos. Hay que mencionar que no se realizará una consulta relacional sino conceptual, de más alto nivel, pero sin llegar a implementar una consulta en lenguaje natural.
	
	Se implementarán, posteriormente, el análisis de métricas del grafo conceptual constituído, lo cual dará lugar a la detección de patrones que no son evidentes o detectables mediante la simple consulta de los datos. Se buscarán medidas de centralidad, dispersión, cercanía, camino más corto, más largo, perímetro y otras que sirvan para caracteriza la información del grafo o un subconjunto de la misma.
	
	Finalmente se elaborarán las conclusiones, necesarias para determinar si la utilización de un grafo conceptual es un mecanismo válido y eficiente para la representación de información heterogénea para detección de patrones, y se enunciarán los trabajos futuros que surjan a partir de dicho análisis.

	
	\section{Cronograma}
	\begin{figure}[H]
%	\begin{center}
		\begin{ganttchart}[
			hgrid,
			vgrid,
			% x unit=0.6cm,
			bar/.append style={draw=Black, fill=RoyalBlue!75},
			time slot format=isodate-yearmonth,
			time slot unit=month,
			newline shortcut=true,
			bar label node/.append style={align=right}
			]{2018-09}{2019-12}
			\gantttitlecalendar{year, month} \\
			\ganttbar{Estudio bibliográfico}{2018-09}{2018-11} \\
			\ganttbar{Análisis de DB\ganttalignnewline de Grafos}{2018-10}{2018-12} \\
			\ganttbar{Análisis de fuentes\ganttalignnewline de información}{2019-01}{2019-03}\\
			\ganttbar{Definición de atributos}{2019-03}{2019-05}\\
			\ganttbar{Modelado de relaciones}{2019-05}{2019-07}\\
			\ganttbar{Algoritmos de\ganttalignnewline adquisición de datos}{2019-06}{2019-08}\\
			\ganttbar{Algoritmos de\ganttalignnewline búsqueda}{2019-07}{2019-10}\\
			\ganttbar{Análisis de métricas}{2019-10}{2019-11}\\
			\ganttbar{Escritura de Tesis}{2019-09}{2019-11}\\
			\ganttmilestone{Defensa de Tesis}{2019-12}
		\end{ganttchart}
%	\end{center}
\end{figure}

	
	\section{Condiciones institucionales}
		Gran parte del trabajo mencionado en el presente documento se puede hacer sin el auxilio de ningún tipo de equipamiento especializado o de software propietario. Aún así, es probable que el acceso a algunas fuentes de datos, de caracter gubernamental, puede requerir de ciertas autorizaciones que será necesario gestionar por parte de la Universidad.
	
	El resumen de las necesidades de hardware, software y acceso se encuentra en la tabla \ref{tabla-necesidades}:
	
	\begin{table}[H]
		\centering
		\begin{tabular}{|p{3cm}|p{8cm}|c|}
			\hline
			\textbf{Etapa} & \textbf{Necesidades} & \textbf{Acceso} \\
			\hline
			Estudio bibliográfico & Internet, WebSites de búsqueda de papers y material bibliográfico & Libre\\
			\hline
			Modelado de DB & Base de datos de grafos\tablefootnote{Las bases de datos deben incluir sus correspondientes herramientas de consulta y administración} & Libre (Mandatorio)\\
			\hline
			Acceso a información & Internet, Bases de datos abiertas, Websites de noticias, Repositorios de datos de gobierno & Libre\tablefootnote{El acceso a datos de gobierno puede requerir permisos especiales o firma de convenios entre la Universidad Tecnológica Nacional y el organismo que corresponda}\\
			\hline
			Redacción de tesis & Notebook, Herramientas de redacción (\LaTeX, paquetes necesarios), Herramientas de corrección, Graficadores & Propiedad del tesista\\
			\hline
			Prototipo experimental & Herramientas de desarrollo (Java, Netbeans, Eclipse o similar) & Libre\\
			\hline
			Todo el desarrollo del trabajo & Ubicación física, aula u otro con acceso a pizarra y archivo & UTN LIS\tablefootnote{Laboratorio de Investigación de Software}\\
			\hline
		\end{tabular}
		\caption{Condiciones institucionales para el desarrollo de la tesis}
		\label{tabla-necesidades}
	\end{table}
	


	\section{Referencias y Bibliografía}
	
	\unsure[inline]{GIRO: Bien. Debería ampliarse.}
	
	\defbibnote{bibnote}{El presente material bibliográfico se ha utilizado como material de estudio pero no se ha citado directamente en el texto.}
	
	% Material CITADO, entra bajo el titulo REFERENCIAS
	\printbibliography[category=cited,title={Referencias}]
	
	% Material NO CITADO, entra bajo el titulo BIBLIOGRAFIA ADICIONAL
	\printbibliography[title={Bibliografía adicional},prenote=bibnote,notcategory=cited,resetnumbers=true,omitnumbers=true]
		
	\ifdefined\borrador
	
	\section*{Notas e Ideas}
	\todo[inline]{Completar esta sección}
	\todo[inline]{Ver alternativas de formato, templates \LaTeX{}}
	
	%\url{maestria@groppo.com.ar}
	
	\subsection*{Tesis de maestría}
	El objeto del trabajo es poder establecer una estructura que permita almacenar y relacionar datos obtenidos de fuentes que no son consistentes ni en su estructura ni en el tipo de información que contienen.
	
	Esto puede incluir fuentes periodísticas, bases de datos gubernamentales, información gráfica (fotos, imágenes), información geográfica, infraestructura (carreteras, controles aduaneros, puertos, etc.), factores climáticos o ambientales, información demográfica, etc.
	
	Esa información debe poder registrarse y correlacionarse con elementos asociados, cualquiera sea el tipo de información de origen o de destino.
	
	La información debe tener asociados atributos genéricos, aplicables a cualquier tipo de datos, y atributos específicos de acuerdo a la información que contenga.
	
	La información debe poder consultarse en todo momento y ser mantenida en tiempo real. Se deben poder hacer consultas sobre cualquier atributo, ya sea genérico o específico y se deben poder obtener todas las relaciones que enlazan los distintos conceptos.
	
	El objetivo final es poder detectar patrones entre esos datos y analizar distintos tipos de relaciones, no evidentes, entre las fuentes de datos.
	
	\subsection*{Tesis de Especialidad}
	La tesis de especialidad debe contener todo el análisis previo que permita evaluar si las bases de datos de grafos son un mecanismo eficiente y recomendable para el almacenamiento de información heterogénea.
	
	El trabajo debe explorar el concepto de grafo, las capacidades de este método de almacenamiento, las alternativas disponibles en productos que implementen este paradigma, la experiencias previas, si las hubiera, y los casos en que se esté utilizando un enfoque similar, de ser posible, con los resultados obtenidos.
	
	\subsection*{Tesis de doctorado}
	Consideremos una red conformada por drones (aéreos, terrestres, marinos o cualquier combinación) en donde los nodos son los dispositivos y los arcos están compuestos por los nexos de comunicación entre ellos. Este modelo se puede entender conceptualmente como un grafo.
	Consideremos que los dispositivos tienen distintas fuentes de información que suministran diferentes tipos de valores, como por ejemplo:
	
	\begin{itemize}
		\item Posición GPS
		\item Sensores de altitud
		\item Sensores de condiciones ambientales
		\item Imágenes
		\item Sensores magnéticos
		\item etc.
	\end{itemize}
	
	El trabajo pretende utilizar lo planteado en la tesis de maestría, con respecto al reconocimiento de patrones en base a fuentes de información heterogéneas. Se busca extrapolar las técnicas que permiten detectar patrones en un grafo dirigido con fuentes de datos variados, a un grafo compuesto por drones como vértices y nexos de comunicación como aristas, considerando los distintos tipos de sensores como fuentes de información.
	
	El objetivo es que un conjunto de dispositivos autónomos reconozcan un patrón general a partir de información parcial detectada por cada uno de ellos. Es de destacar que ninguno de los drones, por si solo, debe tener toda la información disponible.
	
	\section*{Trabajo y evaluación final de HDT}
	\begin{itemize}
		\item Se cumplimentara presentando un borrador del plan de tesis pedido para finalizar la carrera
		\item Recomendable que el tema sea el previsto para la tesis de posgrado
		\item Presentación individual
		\item Alcance
		\begin{enumerate}
			\item Título
			\item Justificación del tema elegido
			\item Fundamentación
			\item Objetivos
			\item Metodología
			\item Plan de trabajo
			\item Condiciones institucionales
			\item Referencias
			\item Bibliografía 
		\end{enumerate}
		\item Extensión: De 4 a 6 carillas. Además la carátula.
		\item Formato acorde a la Circular Nro 2 
		\item Presentación
		\begin{enumerate}
			\item Exposición pública
			\item Duración 10 minutos		
		\end{enumerate}
	\end{itemize}
	
	\listoftodos[Pendientes]
	
	\fi

\end{document}
