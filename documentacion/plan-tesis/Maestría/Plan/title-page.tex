\thispagestyle{empty}
\begin{center}
	\begin{figure}[h]
		\begin{subfigure}{0.5\textwidth}
			\centering
			\includegraphics[width=5cm,left]{./images/utn-black}
		\end{subfigure}
		~
		\begin{subfigure}{0.5\textwidth}
			\centering
			\includegraphics[scale=0.8,right]{./images/MISI3}
		\end{subfigure}
	\end{figure}
	\vspace{1cm}
	\begin{minipage}{0.90\linewidth}
		\centering
		%University logo
		
		%\rule{0.4\linewidth}{0.15\linewidth}\par
		%\vspace{3cm}
		%Thesis title
		{\uppercase{\Large \textbf{Utilización de bases de datos de grafos para la detección de patrones en cúmulos estelares}\par}}
		\vspace{2cm}
		%Author's name
		{\Large Martín Gustavo Casatti\par}
		\vspace{2cm}
		%Degree
		{\Large Plan de Tesis para el Título de Magister en Ingeniería de Sistemas\par}
		\vspace{2cm}
		%Date
		{\Large \DTMMonthname{\the\month} de \the\year}
		\vspace{1cm}
	\end{minipage}
\end{center}
\noindent El presente trabajo se realiza como parte de los requerimientos para la obtención del título de Magíster en Ingeniería de Sistemas, tal como dicta la normativa de la Universidad Tecnológica Nacional.\\

\noindent \hfill\textit{\Today} \\% Printing/edition date
\clearpage