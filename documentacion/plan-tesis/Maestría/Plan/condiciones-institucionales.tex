	Gran parte del trabajo mencionado en el presente documento se puede hacer sin el auxilio de ningún tipo de equipamiento especializado o de software propietario. Aún así, es probable que el acceso a algunas fuentes de datos, de caracter gubernamental, puede requerir de ciertas autorizaciones que será necesario gestionar por parte de la Universidad.
	
	El resumen de las necesidades de hardware, software y acceso se encuentra en la tabla \ref{tabla-necesidades}:
	
	\begin{table}[H]
		\centering
		\begin{tabular}{|p{3cm}|p{8cm}|c|}
			\hline
			\textbf{Etapa} & \textbf{Necesidades} & \textbf{Acceso} \\
			\hline
			Estudio bibliográfico & Internet, WebSites de búsqueda de papers y material bibliográfico & Libre\\
			\hline
			Modelado de DB & Base de datos de grafos\tablefootnote{Las bases de datos deben incluir sus correspondientes herramientas de consulta y administración} & Libre (Mandatorio)\\
			\hline
			Acceso a información & Internet, Bases de datos abiertas, Websites de noticias, Repositorios de datos de gobierno & Libre\tablefootnote{El acceso a datos de gobierno puede requerir permisos especiales o firma de convenios entre la Universidad Tecnológica Nacional y el organismo que corresponda}\\
			\hline
			Redacción de tesis & Notebook, Herramientas de redacción (\LaTeX, paquetes necesarios), Herramientas de corrección, Graficadores & Propiedad del tesista\\
			\hline
			Prototipo experimental & Herramientas de desarrollo (Java, Netbeans, Eclipse o similar) & Libre\\
			\hline
			Todo el desarrollo del trabajo & Ubicación física, aula u otro con acceso a pizarra y archivo & UTN LIS\tablefootnote{Laboratorio de Investigación de Software}\\
			\hline
		\end{tabular}
		\caption{Condiciones institucionales para el desarrollo de la tesis}
		\label{tabla-necesidades}
	\end{table}
	
