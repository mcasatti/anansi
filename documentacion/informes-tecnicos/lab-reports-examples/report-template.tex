\documentclass[a4paper,10pt]{article}
%----------------------------------------------------------------------------------------
%	DEFINE EL FLAG BORRADOR para imprimir, por ejemplo, la lista de TO-DO
%----------------------------------------------------------------------------------------
\def\borrador{}

%\usepackage[]{fontspec}
%\setmainfont{Times New Roman}
\usepackage[]{newtx}
\usepackage[spanish]{babel}

% ======================================================
% GESTION DE COLORES Y GRAFICOS
% ======================================================
\usepackage[dvipsnames,svgnames]{xcolor}
\usepackage{graphicx}
% ======================================================
% MEJORA LA LEGIBILIDAD GENERAL DEL DOCUMENTO Y LAYOUT
% ======================================================
% Microtype tiene que preceder a Ragged2e
\usepackage[babel=true,expansion=true,protrusion=true]{microtype}
% Ragged2e tiene que ir despues de microtype para funcionar correctamente sino cambia las alineaciones y no quedan bien
\usepackage{ragged2e}

% ======================================================
% MARCAS DE AGUA
% ======================================================
\usepackage{draftwatermark}
% ======================================================
% PERMITE INCORPORAR HIPERVINCULOS (internos y URL externas)
% ======================================================
\usepackage[
	colorlinks,
	urlcolor = blue,
	linkcolor = black,
	citecolor = blue
	%pdfborder = {0 0 0}
]{hyperref}


% ======================================================
% MEJORES LISTAS DE ENUMERACION
% ======================================================
\usepackage{enumitem}
% Para poder tachar texto con \uwave{}, uwave{}, sout{}
\usepackage[normalem]{ulem} % El parametro normalem es para no alterar el emphasis (italica)

% ======================================================
% TODOLIST
% ======================================================
\setlength{\marginparwidth}{2cm}
\usepackage[colorinlistoftodos,textsize=tiny]{todonotes}
\ifdefined\borrador
	\newcommand{\revisar}[1]{\todo[backgroundcolor=yellow!50]{\textbf{Revisar:} #1}}
	\newcommand{\ampliar}{\todo[backgroundcolor=blue!30]{Ampliar}}
	\newcommand{\revisarcita}{\colorbox{red}{\textcolor{white}{\ CITA\ }}}
	\newcommand{\sincita}{\colorbox{red}{\textcolor{white}{\ CITA\ }}}
	\newcommand{\faltafigura}{\missingfigure{Falta figura\ldots}}
	\newcommand{\citaurl}{\todo[backgroundcolor=green!50]{URL}}
\else
	\newcommand{\revisar}[1]{}
	\newcommand{\ampliar}{}
	\newcommand{\revisarcita}{}
	\newcommand{\sincita}{}
	\newcommand{\faltafigura}{}
	\newcommand{\citaurl}{}
\fi

\usepackage{csquotes}

% ======================================================
% GANTT
% ======================================================
\usepackage{pgfgantt}
% ======================================================
% FLOATS y REORDENAMIENTO DE TEXTO ALREDEDOR DE FIGURAS
% ======================================================
\usepackage{float}
\usepackage{wrapfig}

\begin{document}

% Title Page
\title{Titulo}
\author{Autor/es \\ e-mail: nnnnnn@gmail.com}
\date{\today}
\maketitle
%\tableofcontents
%\newpage

\begin{abstract}
	Resumen general.
\end{abstract}

\section*{Foreword}
When writing a scientific report it is very important to think carefully how to organize it.

Most reports and scientific papers follow the so called IMRAD structure,that is they are subdivided in four sections: \textbf{I}ntroduction, \textbf{M}ethods, \textbf{R}esults \textbf{A}nd \textbf{D}iscussion.

This is a well-tried format and an efficient way of writing a report, it is highly recommended that you stick to it: 

\begin{quote}
	The goal of a report or a scientific paper is not to impress the readers by poetic language but to transfer facts and new insights as clearly as possible.
\end{quote}

More importantly structuring your paper helps you understand more about the topic you are examining.

\section{Introducción}

\large{Ejemplo de código fuente}

\definecolor{armygreen}{rgb}{0.29, 0.33, 0.13}
\lstset{language=Java,
	tabsize=3,
	showstringspaces=false,
	numbers=left,
	basicstyle=\ttfamily,
	keywordstyle=\color{black}\ttfamily\bfseries,
	stringstyle=\color{armygreen}\ttfamily,
	commentstyle=\color{gray}\ttfamily,
	morecomment=[l][\color{gray}]{\#}
}
\begin{lstlisting}[title={Prueba de titulo},caption={Prueba de caption},captionpos=b,frame=shadowbox]
#include<stdio.h>
#include<iostream>
// A comment
int main(void)
{
printf("Hello World\n");
return 0;
}
\end{lstlisting}

\section{Métodos}

\large{Ejemplo de dibujo de un grafo}

\begin{tikzpicture}
\GraphInit[vstyle=Normal];
\draw[help lines] (0,0) grid (6,3);
\SetGraphUnit{2}
\SetVertexNormal[MinSize=35pt]
\tikzset{EdgeStyle/.style = {->,bend left=20,-latex}}
\tikzset{newstyle/.style = {->,bend left=20,-latex,dashed}}


\Vertex [Math,L=C_1]{C1}
\NOEA [Math,L=C_2](C1) {C2}
\SOEA [Math,L=C_{n-1}](C2) {C3}
\NOEA [Math,L=C_n](C3) {C4}

\Edge[label=$R_{1,2}$,style={above,pos=0.3}](C1)(C2)
\Edge[label=$R_{2,3}$,style={below,pos=0.3,dashed}](C2)(C3)
\Edge[label=$R_{n-1,n}$,style={below,pos=0.7}](C3)(C4)
\end{tikzpicture}

\section{Resultados}

\section{Discusión}

\section{Referencias y bibliografía}

\large{Ejemplo de referencias embebidas}

Para no utilizar un archivo separado. \cite{Simpson}

\begin{thebibliography}{50}
	\bibitem{orientdb} 
	OrientDB, 
	\url{http://orientdb.com/}
	
	\bibitem{Simpson} 
	Homer J. Simpson. \textsl{Mmmmm...donuts}.
	Evergreen Terrace Printing Co., Springfield, SomewhereUSA, 1998
\end{thebibliography}

%----------------------------------------------------------------------------------------
%	LISTA DE TO-DO (CONDICIONAL, DEPENDE DEL FLAG \borrador)
%----------------------------------------------------------------------------------------
\ifdefined\borrador
\section*{Pendientes}
\listoftodos[Pendientes]
\fi

\end{document}          
