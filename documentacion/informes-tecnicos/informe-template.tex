\documentclass[a4paper,headsepline,footsepline,draft=false]{scrartcl}

%\usepackage[]{fontspec}
%\setmainfont{Times New Roman}
\usepackage[]{newtx}
\usepackage[spanish]{babel}

% ======================================================
% GESTION DE COLORES Y GRAFICOS
% ======================================================
\usepackage[dvipsnames,svgnames]{xcolor}
\usepackage{graphicx}
% ======================================================
% MEJORA LA LEGIBILIDAD GENERAL DEL DOCUMENTO Y LAYOUT
% ======================================================
% Microtype tiene que preceder a Ragged2e
\usepackage[babel=true,expansion=true,protrusion=true]{microtype}
% Ragged2e tiene que ir despues de microtype para funcionar correctamente sino cambia las alineaciones y no quedan bien
\usepackage{ragged2e}

% ======================================================
% MARCAS DE AGUA
% ======================================================
\usepackage{draftwatermark}
% ======================================================
% PERMITE INCORPORAR HIPERVINCULOS (internos y URL externas)
% ======================================================
\usepackage[
	colorlinks,
	urlcolor = blue,
	linkcolor = black,
	citecolor = blue
	%pdfborder = {0 0 0}
]{hyperref}


% ======================================================
% MEJORES LISTAS DE ENUMERACION
% ======================================================
\usepackage{enumitem}
% Para poder tachar texto con \uwave{}, uwave{}, sout{}
\usepackage[normalem]{ulem} % El parametro normalem es para no alterar el emphasis (italica)

% ======================================================
% TODOLIST
% ======================================================
\setlength{\marginparwidth}{2cm}
\usepackage[colorinlistoftodos,textsize=tiny]{todonotes}
\ifdefined\borrador
	\newcommand{\revisar}[1]{\todo[backgroundcolor=yellow!50]{\textbf{Revisar:} #1}}
	\newcommand{\ampliar}{\todo[backgroundcolor=blue!30]{Ampliar}}
	\newcommand{\revisarcita}{\colorbox{red}{\textcolor{white}{\ CITA\ }}}
	\newcommand{\sincita}{\colorbox{red}{\textcolor{white}{\ CITA\ }}}
	\newcommand{\faltafigura}{\missingfigure{Falta figura\ldots}}
	\newcommand{\citaurl}{\todo[backgroundcolor=green!50]{URL}}
\else
	\newcommand{\revisar}[1]{}
	\newcommand{\ampliar}{}
	\newcommand{\revisarcita}{}
	\newcommand{\sincita}{}
	\newcommand{\faltafigura}{}
	\newcommand{\citaurl}{}
\fi

\usepackage{csquotes}

% ======================================================
% GANTT
% ======================================================
\usepackage{pgfgantt}
% ======================================================
% FLOATS y REORDENAMIENTO DE TEXTO ALREDEDOR DE FIGURAS
% ======================================================
\usepackage{float}
\usepackage{wrapfig}
\usepackage[sfdefault,scaled=1]{FiraSans}
\usepackage{courier}

\renewcommand{\lstlistingname}{Listado}
\lstset{
	language=sql,
	numbers=left,
	numberstyle=\tiny,
	breaklines=true,
	frame=shadowbox,
	captionpos=b
}


% Font = Times New Roman
%\usepackage{times}
%\renewcommand*\rmdefault{ppl}
%\renewcommand*\sfdefault{pag}
%\renewcommand{\familydefault}{\sfdefault}

\clearpairofpagestyles
\ohead{Informe: \today}
\ofoot{Página \pagemark}

%\ifoot{Pie}
%\ofoot{Página \pagemark}

%----------------------------------------------------------------------------------------
%	DEFINE EL FLAG BORRADOR para imprimir, por ejemplo, la lista de TO-DO
%----------------------------------------------------------------------------------------

\def\borrador{}

\begin{document}

% Title Page
\title{--TITULO--}
\author{Ing. Martín G. Casatti \\ e-mail: mcasatti@frc.utn.edu.ar}
\date{\today}
\maketitle
%\tableofcontents
%\newpage
\thispagestyle{headings}

\begin{abstract}
	--ABSTRACT--
\end{abstract}

%\ifdefined\borrador
%\section*{Foreword}
%When writing a scientific report it is very important to think carefully how to organize it.
%
%Most reports and scientific papers follow the so called IMRAD structure,that is they are subdivided in four sections: \textbf{I}ntroduction, \textbf{M}ethods, \textbf{R}esults \textbf{A}nd \textbf{D}iscussion.
%
%This is a well-tried format and an efficient way of writing a report, it is highly recommended that you stick to it: 
%
%\begin{quote}
%	The goal of a report or a scientific paper is not to impress the readers by poetic language but to transfer facts and new insights as clearly as possible.
%\end{quote}
%
%More importantly structuring your paper helps you understand more about the topic you are examining.
%\fi

\section{Introducción}
% ********************************************************************************
% Introduction ¿Porque has comenzado?¿Qué motiva el trabajo?¿Cuál es la pregunta?
% ********************************************************************************

%\begin{lstlisting}[title={Prueba de titulo},caption={Prueba de caption},captionpos=b,frame=shadowbox]
%#include<stdio.h>
%#include<iostream>
%// A comment
%int main(void)
%{
%	printf("Hello World\n");
%	return 0;
%}
%\end{lstlisting}
%\fi

\section{Métodos}
% ********************************************************************************
% Methods ¿Que has hecho?¿Qué buscabas? ¿Cómo lo has hecho?
% ********************************************************************************

\section{Resultados}
% ********************************************************************************
% Results ¿Que has descubierto?
% ********************************************************************************

\section{Discusión}
% ********************************************************************************
% Discussion ¿Que significa todo eso?
% ********************************************************************************

\section{Referencias y bibliografía}
\begin{thebibliography}{50}
	\bibitem{dropbox} 
		DropBox para descarga de archivos, 
		\url{https://www.dropbox.com/sh/kniy70dwlx5s1ow/AACySwi4pJsctAstrZf-mK3Wa?dl=0}
	\bibitem{topcat} 
		TOPCAT (herramienta de gestión de tablas de texto), 
		\url{http://www.star.bristol.ac.uk/~mbt/topcat/#starjava}
\end{thebibliography}

%----------------------------------------------------------------------------------------
%	LISTA DE TO-DO (CONDICIONAL, DEPENDE DEL FLAG \borrador)
%----------------------------------------------------------------------------------------
\ifdefined\borrador
\section*{Pendientes}
	\listoftodos
\fi

\end{document}          
