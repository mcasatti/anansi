\documentclass[a4paper,headsepline,footsepline,draft=false]{scrartcl}

% Required for specifying colors by name
\usepackage[usenames,dvipsnames,svgnames,table]{xcolor} 
% Define the orange color used for highlighting throughout the book
\definecolor{ocre}{RGB}{243,102,25} 
\usepackage[export]{adjustbox}
\usepackage{subcaption}
\usepackage{float}
% Permite utilizar encabezados y pies de pagina simples con KOMA-Script
\usepackage{scrlayer-scrpage}
% Page margins
\usepackage[top=3cm,bottom=3cm,left=3cm,right=3cm,headsep=10pt,a4paper]{geometry} 
\usepackage[document]{ragged2e}
% Required for including pictures
\usepackage{graphicx} 
% Inserts dummy text
\usepackage{lipsum} 
%\usepackage[english]{babel} % English language/hyphenation
\usepackage[spanish,es-noquoting]{babel} % English language/hyphenation
% Customize lists
\usepackage{enumitem} 
% Resaltado de sintaxis en listados de código fuente
\usepackage{listings}
% Required for nicer horizontal rules in tables
\usepackage{booktabs} 
% Mejor manejo de tablas
\usepackage{tabularx} 
% Manejo de hipervinculos
\usepackage[hidelinks]{hyperref}
%Manejo de referencias y bibliografía
%\usepackage
%[
%	%style=alphabetic,
%	style=numeric,
%	citestyle=numeric,
%	sorting=none,
%	%sortcites=true,
%	%autopunct=true,
%	babel=hyphen,
%	hyperref=true,
%	abbreviate=false,
%	backref=true,
%	%backend=biblatex,
%	defernumbers=false
%]
%{biblatex}
% Permite sacar parte del codigo a un archivo externo. Util para referenciar bibliografía embebida
%\usepackage{filecontents}
%----------------------------------------------------------------------------------------
%	ALGORITMOS
%----------------------------------------------------------------------------------------
\usepackage{algorithm2e}
% Slightly tweak font spacing for aesthetics
%\usepackage{microtype} 
% Required for including letters with accents
\usepackage[utf8]{inputenc} 
% Use 8-bit encoding that has 256 glyphs
\usepackage[T1]{fontenc} 
% Agregar fuentes monoespaciadas bold
\usepackage{bold-extra}
% Using Courier font
%\renewcommand{\ttdefault}{pcr}
%----------------------------------------------------------------------------------------
%	DEFINICION DE DIVERSOS TIPOS DE TO-DO
%----------------------------------------------------------------------------------------
\usepackage[colorinlistoftodos,prependcaption,textsize=tiny]{todonotes}
\usepackage{tablefootnote}
\usepackage{xargs}

\newcommandx{\revisar}[2][1=]{\todo[linecolor=red,backgroundcolor=red!25,bordercolor=red,#1]{#2}}
\newcommandx{\cambiar}[2][1=]{\todo[linecolor=blue,backgroundcolor=blue!25,bordercolor=blue,#1]{#2}}
\newcommandx{\info}[2][1=]{\todo[linecolor=green,backgroundcolor=green!25,bordercolor=green,#1]{#2}}
\newcommandx{\mejorar}[2][1=]{\todo[linecolor=gray,backgroundcolor=Plum!25,bordercolor=gray,#1]{#2}}
\newcommandx{\faltafigura}[2][1=]{\missingfigure[#1,figcolor=white]{#2}}
\newcommandx{\nomostrar}[2][1=]{\todo[disable,#1]{#2}}

% Package para creacion de grafos simplificados
\usepackage{tkz-graph}
% Required for drawing custom shapes
\usepackage{tikz} 
\usetikzlibrary{
	arrows,
	decorations.pathmorphing,
	backgrounds,
	positioning,
	fit,
	petri,
	quotes,
	% Este paquete produce errores usado en conjunto con tkz-graph
	%babel,	
	arrows.meta,
	decorations.pathreplacing,
	shapes
}

%----------------------------------------------------------------------------------------
%	CONFIGURACION DE LOS LISTADOS DE CODIGO FUENTE
%----------------------------------------------------------------------------------------
\definecolor{armygreen}{rgb}{0.29, 0.33, 0.13}
\lstset{language=Java,
	tabsize=3,
	showstringspaces=false,
	numbers=left,
	basicstyle=\ttfamily,
	keywordstyle=\color{black}\ttfamily\bfseries,
	stringstyle=\color{armygreen}\ttfamily,
	commentstyle=\color{gray}\ttfamily,
	morecomment=[l][\color{gray}]{\#}
}

\usepackage[sfdefault,scaled=1]{FiraSans}
\usepackage{courier}

\renewcommand{\lstlistingname}{Listado}
\lstset{
	language=sql,
	numbers=left,
	numberstyle=\tiny,
	breaklines=true,
	frame=shadowbox,
	captionpos=b
}


% Font = Times New Roman
%\usepackage{times}
%\renewcommand*\rmdefault{ppl}
%\renewcommand*\sfdefault{pag}
%\renewcommand{\familydefault}{\sfdefault}

\clearpairofpagestyles
\ohead{Informe: \today}
\ofoot{Página \pagemark}

%\ifoot{Pie}
%\ofoot{Página \pagemark}

%----------------------------------------------------------------------------------------
%	DEFINE EL FLAG BORRADOR para imprimir, por ejemplo, la lista de TO-DO
%----------------------------------------------------------------------------------------

\def\borrador{}

\begin{document}

% Title Page
\title{--TITULO--}
\author{Ing. Martín G. Casatti \\ e-mail: mcasatti@frc.utn.edu.ar}
\date{\today}
\maketitle
%\tableofcontents
%\newpage
\thispagestyle{headings}

\begin{abstract}
	--ABSTRACT--
\end{abstract}

%\ifdefined\borrador
%\section*{Foreword}
%When writing a scientific report it is very important to think carefully how to organize it.
%
%Most reports and scientific papers follow the so called IMRAD structure,that is they are subdivided in four sections: \textbf{I}ntroduction, \textbf{M}ethods, \textbf{R}esults \textbf{A}nd \textbf{D}iscussion.
%
%This is a well-tried format and an efficient way of writing a report, it is highly recommended that you stick to it: 
%
%\begin{quote}
%	The goal of a report or a scientific paper is not to impress the readers by poetic language but to transfer facts and new insights as clearly as possible.
%\end{quote}
%
%More importantly structuring your paper helps you understand more about the topic you are examining.
%\fi

\section{Introducción}
% ********************************************************************************
% Introduction ¿Porque has comenzado?¿Qué motiva el trabajo?¿Cuál es la pregunta?
% ********************************************************************************

%\begin{lstlisting}[title={Prueba de titulo},caption={Prueba de caption},captionpos=b,frame=shadowbox]
%#include<stdio.h>
%#include<iostream>
%// A comment
%int main(void)
%{
%	printf("Hello World\n");
%	return 0;
%}
%\end{lstlisting}
%\fi

\section{Métodos}
% ********************************************************************************
% Methods ¿Que has hecho?¿Qué buscabas? ¿Cómo lo has hecho?
% ********************************************************************************

\section{Resultados}
% ********************************************************************************
% Results ¿Que has descubierto?
% ********************************************************************************

\section{Discusión}
% ********************************************************************************
% Discussion ¿Que significa todo eso?
% ********************************************************************************

\section{Referencias y bibliografía}
\begin{thebibliography}{50}
	\bibitem{dropbox} 
		DropBox para descarga de archivos, 
		\url{https://www.dropbox.com/sh/kniy70dwlx5s1ow/AACySwi4pJsctAstrZf-mK3Wa?dl=0}
	\bibitem{topcat} 
		TOPCAT (herramienta de gestión de tablas de texto), 
		\url{http://www.star.bristol.ac.uk/~mbt/topcat/#starjava}
\end{thebibliography}

%----------------------------------------------------------------------------------------
%	LISTA DE TO-DO (CONDICIONAL, DEPENDE DEL FLAG \borrador)
%----------------------------------------------------------------------------------------
\ifdefined\borrador
\section*{Pendientes}
	\listoftodos
\fi

\end{document}          
